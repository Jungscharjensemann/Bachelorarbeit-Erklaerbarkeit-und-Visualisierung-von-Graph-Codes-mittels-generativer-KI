\begin{usecase}{UC-1.1 Graph Code(s) importieren}
\label{sec3:model:uc-1.1}
    \desc{
        Benutzer klicken in der Arbeitsfläche einen Knopf \enquote{Select Graph Code(s)}.
        Daraufhin öffnet sich ein Filechooser, in welchem Benutzer eine Auswahl von einem oder mehrerer Graph Code Datei(en) treffen können.
        Die ausgewählten Graph Code Datei(en) werden einer Liste in der Arbeitsfläche hinzugefügt und angezeigt.
    }
    \tcbline
    \actors{Benutzer, System}
    \tcbline
    \pre{Keine.}
    \tcbline
    \mainflow{
        \item System zeigt eine Liste in der Arbeitsfläche an.
        \item Benutzer klickt Knopf \enquote{Select Graph Code(s).}
        \item System zeigt Filechooser an.
        \item Benutzer wählt ein oder mehrere Graph Code Datei(en) aus.
        \item System fügt Graph Code Datei(en) der Liste in der Arbeitsfläche zu.
    }
    \tcbline
    \post{Graph Code Datei(en) sind der Liste in der Arbeitsfläche hinzugefügt worden.}
\end{usecase}

\begin{usecase}{UC-1.2 Graph Code(s) entfernen}
\label{sec3:model:uc-1.2}
    \desc{
        Benutzer wählen aus der Liste in der Arbeitsfläche ein oder mehrere Graph Code Datei(en) aus und können über einen Knopf \enquote{Remove selected Graph Code(s)} diese Graph Codes aus der Liste und somit der Arbeitsfläche entfernen.
    }
    \tcbline
    \actors{Benutzer, System}
    \tcbline
    \pre{
        \begin{itemize}
            \item Liste enthält ein oder mehrere Graph Code(s).
            \item Benutzer haben ein oder mehrere Graph Code(s) ausgewählt.
        \end{itemize}
    }
    \tcbline
    \mainflow{
        \item System zeigt eine Liste an Graph Codes in der Arbeitsfläche an.
        \item Benutzer wählt ein oder mehrere Graph Code(s) aus.
        \item Benutzer klickt auf Knopf \enquote{Remove selected Graph Code(s).}
        \item System entfernt ausgewählte Graph Codes aus der Liste in der Arbeitsfläche.
    }
    \tcbline
    \post{Ausgewählte Graph Code Datei(en) sind aus der Liste in der Arbeitsfläche entfernt worden.}
\end{usecase}

\begin{usecase}{UC-1.3 Graph Code(s) auswählen}
\label{sec3:model:uc-1.3}
    \desc{
        Benutzer wählen aus der Liste in der Arbeitsfläche ein oder mehrere Graph Code Datei(en) aus.
        Die Auswahl von Graph Code Dateien dient als Grundlage für Anwendungsfälle, wie \hyperref[sec3:model:uc-1.2]{UC-1.2}, \hyperref[sec3:model:uc-1.5]{UC-1.5} oder \hyperref[sec3:model:uc-1.8]{UC-1.8}.
    }
    \tcbline
    \actors{Benutzer, System}
    \tcbline
    \pre{
        Liste enthält ein oder mehrere Graph Code(s) zum Auswählen.
    }
    \tcbline
    \mainflow{
        \item System zeigt eine Liste an Graph Code(s) in der Arbeitsfläche an.
        \item Benutzer wählen ein oder mehrere Graph Code(s) an.
    }
    \tcbline
    \post{Keine.}
\end{usecase}

\begin{usecase}{UC-1.4 Operation auswählen}
\label{sec3:model:uc-1.4}
    \desc{
        Benutzer wählen aus einem Feld eine auszuführende Operation aus.
        Verfügbare Optionen sind: Vereinigung, Subtraktion, Gemeinsamkeiten, Unterschiede.
        Das Auswählen einer Operation ist die Vorbedingung für den Anwendungsfall \hyperref[sec3:model:uc-1.5]{UC-1.5}, dem Ausführen einer Operation.
    }
    \tcbline
    \actors{Benutzer, System}
    \tcbline
    \pre{Keine.}
    \tcbline
    \mainflow{
        \item System bietet in einem Feld eine Reihe an auszuwählenden Operationen.
        \item Benutzer wählt eine Operation aus.
    }
    \tcbline
    \post{Keine.}
\end{usecase}

\begin{usecase}{UC-1.5 Operation ausführen}
\label{sec3:model:uc-1.5}
    \desc{
        Benutzer klicken auf den Knopf \enquote{Execute}, um die vorher ausgewählte Operation auf den in der Arbeitsfläche ausgewählten Graph Code Datei(en) auszuführen.
        Die vorher ausgewählte Operation wird daraufhin auf den ausgewählten Graph Code Datei(en) ausgeführt.
    }
    \tcbline
    \actors{Benutzer, System}
    \tcbline
    \pre{
        \begin{itemize}
            \item Operation in der Arbeitsfläche ausgewählt.
            \item In der Arbeitsfläche wurden Graph Code Datei(en) ausgewählt.
        \end{itemize}
    }
    \tcbline
    \mainflow{
        \item Benutzer klickt auf Knopf \enquote{Execute}.
        \item System führt vorher ausgewählte Operation auf den ausgewählten Graph Code Datei(en) aus.
    }
    \tcbline
    \post{Erfolgreich ausgeführte Operation auf ausgewählten Graph Code(s).}
\end{usecase}

\begin{usecase}{UC-1.6 Erklärungstyp umschalten}
\label{sec3:model:uc-1.6}
    \desc{
        Benutzer wählen in der Arbeitsfläche über Knöpfe \enquote{Image} oder \enquote{Text} den Typ der Erklärung aus.
        Anhand der ausgewählten Erklärung wird die Benutzerschnittstelle in der Arbeitsfläche für den jeweiligen Erklärungstyp umgeschaltet.
    }
    \tcbline
    \actors{Benutzer, System}
    \tcbline
    \pre{Keine.}
    \tcbline
    \mainflow{
        \item Benutzer wählen über die Knöpfe \enquote{Image} oder \enquote{Text} den Typ der Erklärung aus.
        \item System schaltet auf für Typ spezifische Benutzerschnittstelle um.
    }
    \tcbline
    \post{Benutzerschnittstelle für spezifischen Erklärungstyp umgeschaltet.}
\end{usecase}

\begin{usecase}{UC-1.7 Erweiterte Optionen anpassen}
\label{sec3:model:uc-1.7}
    \desc{
        Benutzer passen in einem dafür vorgesehenen Feld erweiterte Optionen für den Endpunkt, der für die zu generierende Erklärung zuständig ist, an.
        Die anpassbaren Optionen sind abhängig vom jeweiligen Endpunkt bzw. Erklärungstypen.
    }
    \tcbline
    \actors{Benutzer, System}
    \tcbline
    \pre{Keine.}
    \tcbline
    \mainflow{
        \item System bietet, abhängig vom Endpunkt bzw. Erklärungstypen, eine Reihe an anpassbaren Optionen an.
        \item Benutzer passen Optionen nach eigenen Bedürfnissen an.
    }
    \tcbline
    \post{Keine.}
\end{usecase}

\begin{usecase}{UC-1.8 Erklärung generieren}
\label{sec3:model:uc-1.8}
    \desc{
        Benutzer klicken auf einen Knopf \enquote{Generate ...}, um eine Erklärung für eine zuvor ausgewählte Graph Code Datei zu generieren.
        Die generierte Erklärung ist abhängig vom zuvor gewählten Endpunkt bzw. Erklärungstypen.
    }
    \tcbline
    \actors{Benutzer, System}
    \tcbline
    \pre{Graph Code Datei wurde ausgewählt.}
    \tcbline
    \mainflow{
        \item Benutzer klickt auf Knopf \enquote{Generate ...}
        \item System lässt durch Endpunkt Erklärung generieren.
        \item System zeigt Erklärung in einem dafür vorgesehenen Bereich der Benuzterschnittstelle an.
    }
    \tcbline
    \post{Keine.}
    \tcbline
    % Hier im Branchflow zwischen Image und Text differenzieren.
    \branchflow{
        \setcounter{enumi}{2}
        \item[\number\value{enumi}.a] Endpunkt für \enquote{Image} generiert eine visuelle Erklärung bzw. ein Bild.
        \item[\number\value{enumi}.b] Endpunkt für \enquote{Text} generiert eine textuelle Erklärung.
    }
\end{usecase}