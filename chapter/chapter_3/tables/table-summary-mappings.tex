\begin{table}[htb]
    \centering
    \begin{tabularx}{\linewidth}{
            @{}
            >{
                \hsize=0.27\linewidth
                \raggedright\arraybackslash
            }X
            >{
                \hsize=0.73\linewidth
            }X
            @{}
        }
        \toprule
        % First Row
        \multicolumn{1}{
            >{
                \hsize=0.27\linewidth\centering\arraybackslash
            }X
        }{
            \textbf{Forschungsziel vom Typ Modellierung}
        } & \multicolumn{1}{
            >{
                \hsize=0.73\linewidth\centering\arraybackslash
            }X
        }{
            \textbf{Wesentliche Arbeitsergebnisse der Modellierung}
        } \\

        % Lower Rows
        \midrule
        Forschungsziel 1.2: Erklärbarkeit von MMIR mittels generativer KI & Test12 \\
        %Forschungsziel 1.2: Konfiguration und Administration von Processing Flows & In \cref{model-use-cases} wurden Anwendungsfälle identifiziert, dargestellt und textuell beschrieben. Aufbauend auf diesen Anwendungsfällen wurden Mechanismen vorgestellt, mit deren Hilfe assoziierte Klassen von Anwendungsfällen identifiziert wurden. In \cref{model-interaction-ui-pfs} wurden die Interaktionen eines Benutzers mit dem Programm beleuchtet und darauf aufbauend ein Wireframe / Mockup für eine Benutzungsschnittstelle konzipiert. \cref{model-interaction-ui-pfs} behandelt somit die offene Herausforderung \hyperref[state-of-the-art-open-issue-2-1]{\textbf{OH 2.1}}. Des Weiteren wurden in \cref{model-interaction-use-cases-seq} die Interaktionen mit dem Programm, in Kombination mit den textuellen Beschreibungen der Anwendungsfälle, in Sequenzdiagrammen dargestellt. \\
        \midrule
        Forschungsziel 2.2: Integration von generativer KI in das GMAF & Test22\\
        %Forschungsziel 2.2: Integration von Processing Flows & In \cref{model-tech-details-pfs} wurden die technischen Details von Processing Flows dargestellt. Darauf aufbauend wird angesprochen, wie Datenstrukturen konzipiert sein müssen, um Processing Flows repräsentieren zu können. Zudem wird angesprochen, wie Processing Flows validiert werden können. Des Weiteren wird in \cref{model-export-pfs} angesprochen, wie ein Processing Flow exportiert werden kann. \cref{model-export-pfs-gen-xml} behandelt die offene Herausforderung \hyperref[state-of-the-art-open-issue-2-2]{\textbf{OH 2.2}} und spricht an, wie aus der Struktur eines Processing Flows eine XML-Datei generiert werden kann.\\
        \bottomrule
    \end{tabularx}
    \caption{Wesentliche Arbeitsergebnisse der Modellierung}
    \label{model-summary-table}
\end{table}
