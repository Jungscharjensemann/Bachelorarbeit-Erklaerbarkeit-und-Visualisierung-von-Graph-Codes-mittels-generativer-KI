\begingroup
\def\arraystretch{1.1}%
\begin{xltabular}{\linewidth}{
            @{}
            >{
                \hsize=0.2\linewidth
                \raggedright\arraybackslash
            }X
            >{
                \hsize=0.6\linewidth
                \raggedright\arraybackslash
            }X
            >{
                \hsize=0.2\linewidth
            }X
            @{}
        }

        % First Header

        \caption{Tabelle zur Übersicht des aktuellen Arbeitsstands.}
        \label{sec3:model:subsec:summary:table:summary}
        \\

        \toprule
        \multicolumn{3}{
            >{
                    \hsize=\linewidth\centering\arraybackslash
            }X
        }
        {
            \textbf{Forschungsziele}
        } \\ \midrule
        \textbf{FZ / OH} &  \textbf{Beschreibung} & \textbf{Referenz} \\ \midrule

        \endfirsthead

        \toprule
        \multicolumn{3}{
            >{
                    \hsize=\linewidth\centering\arraybackslash
            }X
        }
        {
            \textbf{Forschungsziele}
        } \\ \midrule
        \textbf{FZ / OH} & \textbf{Beschreibung} & \textbf{Referenz} \\ \midrule

        \endhead

        % Lower Rows

        \multicolumn{3}{
            >{
                    \hsize=\linewidth\centering\arraybackslash
            }X
        }
        {
            \textbf{Erklärbarkeit von MMIR mittels generativer KI}
        }
        \\
        \midrule

        FZ 1.1/O
        &
        Recherche zur Erklärbarkeit von MMIR mittels generativer KI
        \\

        &
        Grundlegende Technologien:
        &

        \\

        &
        \tabitem GMAF
        &
        \cref{sec2:sota:subsubsec:gmaf}
        \\

        &
        \tabitem MMFG
        &
        \cref{sec2:sota:subsubsec:mmfg}
        \\

        &
        \tabitem Graph Code
        &
        \cref{sec2:sota:subsubsec:graph-codes}
        \\

        % Offene Herausforderungen aus FZ1/O

        OH 1.1
        &
        Erste offene Herausforderung
        &
        \hyperref[sec2:sota:oi:1.1]{\textbf{OH 1.1}}
        \\


        &
        Systeme generativer KI und ein Überlick über aktuelle Systeme
        &
        \cref{sec2:sota:subsubsec:genai}
        \\

        &
        Diskussion und Auswahl von Systemen
        &
        \cref{sec2:sota:subsubsec:fz1:discussion}
        \\

        OH 1.2
        &
        Zweite offene Herausforderung
        &
        \hyperref[sec2:sota:oi:2.1]{\textbf{OH 2.1}}
        \\

        \midrule

        FZ 1.2/TB
        &
        Modellierung der Erklärbarkeit von MMIR mittels generativer KI
        &

        \\

        &
        Erklärbarkeit durch generative KI
        &
        \cref{sec3:model:subsubsec:explainability-through-genai}
        \\

        &
        $\rightarrow$ Behandlung der ersten offenen Herausforderung \hyperref[sec2:sota:oi:1.1]{\textbf{OH 1.1}}
        &
        \\

        &
        Anwendungsfälle:
        &
        \cref{sec3:model:subsubsec:use-cases}
        \\

        &
        \tabitem Textuelle Beschreibungen
        &
        %\cref{sec3:model:par:textual-desc-use-cases}
        \\

        &
        Wireframes
        &
        \cref{sec3:model:par:wireframe}
        \\

        &
        Mechanismen
        &
        \cref{sec3:model:par:mechanism-use-cases}
        \\

        &
        Sequenzdiagramme
        &
        \cref{sec3:model:par:seq-use-cases}
        \\

        &
        $\rightarrow$ Behandlung der zweiten offenen Herausforderung \hyperref[sec2:sota:oi:1.2]{\textbf{OH 1.2}}
        &
        \\

        \midrule

        FZ 1.3/I
        &
        Implementierung der Erklärbarkeit von MMIR mittels generativer KI
        &

        \\

        \midrule

        FZ 1.4/E
        &
        Evaluierung der Erklärbarkeit von MMIR mittels generativer KI
        &

        \\

        \midrule

        \multicolumn{3}{
            >{
                    \hsize=\linewidth\centering\arraybackslash
            }X
        }
        {
            \textbf{Integration generativer KI in das GMAF}
        }
        \\
        \midrule

        FZ 2.1/O
        &
        Recherche zur Integration generativer KI in das GMAF
        &

        \\


        &
        Aufzeigen der Integrationsmöglichkeiten von:
        &

        \\

        &
        \tabitem Graph Codes
        &
        \cref{sec2:sota:subsubsec:gc-capabilities-integration}
        \\

        &
        Erste offene Herausforderung
        &
        \hyperref[sec2:sota:oi:2.1]{\textbf{OH 2.1}}
        \\

        &
        \tabitem Systemen generativer KI
        &
        \cref{sec2:sota:subsubsec:genai-capabilities-integration}
        \\

        &
        Zweite offene Herausforderung
        &
        \hyperref[sec2:sota:oi:2.2]{\textbf{OH 2.2}}
        \\

        \midrule

        FZ 2.2/TB
        &
        Modellierung der Integration generativer KI in das GMAF
        &

        \\

        &
        Transformation von Graph Codes
        &
        \cref{sec3:model:subsubsec:gc-transformation}
        \\

        &
        \tabitem Transformation des Vokabulars
        &
        \\

        &
        \tabitem Transformation der Matrix
        &
        \\

        &
        \tabitem Anwendung von Graph Code Metriken
        &
        \\

        &
        $\rightarrow$ Behandlung der ersten offenen Herausforderung \hyperref[sec2:sota:oi:2.1]{\textbf{OH 2.1}}
        &
        \\

        &
        Einbindung generativer KI in das GMAF
        &
        \cref{sec3:model:subsubsec:genai-integration}
        \\

        &
        $\rightarrow$ Behandlung der zweiten offenen Herausforderung \hyperref[sec2:sota:oi:2.2]{\textbf{OH 2.2}}
        &
        \\

        \midrule

        FZ 2.3/I
        &
        Implementierung der Integration generativer KI in das GMAF
        &

        \\

        \midrule

        FZ 2.4/E
        &
        Evaluierung der Integration generativer KI in das GMAF
        &

        \\

        \bottomrule
\end{xltabular}
\endgroup
