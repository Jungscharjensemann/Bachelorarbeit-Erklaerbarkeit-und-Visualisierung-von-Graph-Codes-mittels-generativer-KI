\begin{figure}[!ht]
    \centering
    \begin{tikzpicture}
            \begin{umlseqdiag}
                % Actors
                \umlactor[fill=white, scale=0.7]{Benutzer}

                % Objects
                \umlobject[x=0, fill=white]{Benutzer}
                \umlmlobject[x=3.3, fill=white]{rsgc}{\shortstack[c]{Remove\\Selected\\GraphCodes\\Controller}}
                \umlmlobject[x=6, fill=white]{egc}{\shortstack[c]{Editor\\GraphCode}}
                \umlobject[x=8, fill=white]{JList}

                % Calls
                \begin{umlcall}[padding=-2,dt=6.5,op={\shortstack[c]{Remove Selected \\ Graph Codes}}]{Benutzer}{rsgc}
                    \begin{umlcall}[dt=4,op={Gib Liste}, return={Liste}]{rsgc}{egc}
                    \end{umlcall}
                    \begin{umlcall}[dt=6,op={\shortstack[c]{Ausgewählte \\Elemente}}, return={Indizes i der Elemente}]{rsgc}{JList}
                    \end{umlcall}
                    \begin{umlfragment}[type=for(ind : i)]
                        \begin{umlcall}[dt=5,op={\shortstack[c]{Entferne Element \\ an Pos. ind}}]{rsgc}{JList}
                        \end{umlcall}
                    \end{umlfragment}
                \end{umlcall}
            \end{umlseqdiag}
        \end{tikzpicture}
    \caption{UML-Sequenzdiagramm für den Anwendungsfall \hyperref[sec3:model:uc-1.2]{UC-1.2}.}
    \label{sec3:model:par:seq-use-cases:fig:seq-diag-uc-1.2}
\end{figure}
