\begin{figure}[htb]
    \centering
    \resizebox{\textwidth}{!}{
        \begin{tikzpicture}
            \begin{umlseqdiag}
                % Actors
                \umlactor[fill=white, scale=0.7]{Benutzer}

                % Objects
                \umlobject[x=0, fill=white]{Benutzer}
                \umlmlobject[x=3.3, fill=white]{tip}{\shortstack[c]{(Text/Image)Panel}}
                \umlmlobject[x=7, fill=white]{rq}{(...)Request}
                \umlmlobject[x=10, fill=white]{bld}{\shortstack[c]{(...)Request\\Builder}}
                \umlmlobject[x=13, fill=white]{oas}{OpenAiService}


                % Calls
                \begin{umlcall}[padding=0,dt=7,op={\shortstack[c]{Erklärung \\ generieren}}]{Benutzer}{tip}
                    \begin{umlcall}[padding=2.5,op={Erzeuge Anfrage},return={Anfrage}]{tip}{rq}
                    \end{umlcall}
                    \begin{umlcall}[padding=2.5,dt=4,op={Parameter der Anfrage anpassen}, return={Modifizierte Anfrage}]{tip}{bld}
                    \end{umlcall}
                    \begin{umlcall}[padding=2.5,dt=4,op={Anfrage an Endpunkt senden}, return={Ergebnis in Form von (Text/Bild)}]{tip}{oas}
                    \end{umlcall}
                    \begin{umlcallself}[dt=4,op={\shortstack[c]{Ergebnis\\verarbeiten\\/ anzeigen}}]{tip}
                    \end{umlcallself}
                \end{umlcall}
            \end{umlseqdiag}
        \end{tikzpicture}
    }
    \caption{UML-Sequenzdiagramm für den Anwendungsfall \hyperref[sec3:model:uc-1.8]{UC-1.8}.}
    \label{sec3:model:par:seq-use-cases:fig:seq-diag-uc-1.8}
\end{figure}
