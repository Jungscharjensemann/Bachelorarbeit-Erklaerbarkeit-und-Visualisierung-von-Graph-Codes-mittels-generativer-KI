\begin{figure}[!ht]
    \centering
    \resizebox{\textwidth}{!}{
        \begin{tikzpicture}
            \begin{umlseqdiag}
                % Actors
                \umlactor[fill=white, scale=0.7]{Benutzer}

                % Objects
                \umlobject[x=0, fill=white]{Benutzer}
                \umlmlobject[x=3.3, fill=white]{sgc}{\shortstack[c]{Select\\GraphCodes\\Controller}}
                \umlobject[x=6, fill=white]{JFileChooser}
                \umlmlobject[x=9, fill=white]{egc}{\shortstack[c]{Editor\\GraphCode}}
                \umlmlobject[x=12, fill=white]{gcle}{\shortstack[c]{GraphCode\\ListElement}}
                \umlmlobject[x=15, fill=white]{gcio}{GraphCodeIO}

                % Calls
                \begin{umlcall}[padding=-1pt,dt=6.5,op={\shortstack[c]{Select \\ Graph Codes}}]{Benutzer}{sgc}
                    \begin{umlcall}[op={zeige Dialog}, return={Zustand}]{sgc}{JFileChooser}
                    \end{umlcall}
                    \begin{umlcall}[dt=4,op={Gib Liste}, return={Liste}]{sgc}{egc}
                    \end{umlcall}
                    \begin{umlfragment}[type=if, label={\shortstack[c]{Zustand \\ \textit{öffnen}}}]
                        \begin{umlcall}[padding=2.5pt,dt=5, op={\shortstack[c]{Ausgewählte \\ Datei(en)}}, return={Datei(en)}]{sgc}{JFileChooser}
                        \end{umlcall}
                        \begin{umlfragment}[type=for (files)]
                            \begin{umlcall}[padding=2.5, op={Erzeuge aus Datei}, return={GraphCodeListElement (gcle)}]{sgc}{gcle}
                                \begin{umlcall}[padding=2.5pt,op={\shortstack[c]{Graph Code \\ Datei lesen}}, return={Graph Code}]{gcle}{gcio}
                                \end{umlcall}
                            \end{umlcall}
                            \begin{umlcall}[op={gcle der Liste hzf.}]{sgc}{egc}
                            \end{umlcall}
                        \end{umlfragment}
                    \end{umlfragment}
                \end{umlcall}
            \end{umlseqdiag}
        \end{tikzpicture}
    }
    \caption{Sequenzdiagramm für den Anwendungsfall \hyperref[sec3:model:uc-1.1]{UC-1.1}.}
    \label{sec3:model:par:seq-use-cases:fig:seq-diag-uc-1.1}
\end{figure}
