\begin{figure}[!htb]
    \centering
    \resizebox{\textwidth}{!}{
        \begin{tikzpicture}
            \def\corner{5mm};
             \def\cornerradius{0.2mm};
             \def\lwidth{0.3mm};
             \def\h{3.5cm};
             \def\w{4.8cm};
             \def\nline{10};
             \def\iconmargin{1mm};
             \def\topmargin{2mm};
             \foreach[count=\i] \filename in {\textbf{Schmetterling}}
             {
               \coordinate (nw) at ($(0,0)$);
               \coordinate (ne0) at ($(nw) + (\w, 0)$);
               \coordinate (ne1) at ($(ne0) - (\corner, 0)$);
               \coordinate (ne2) at ($(ne0) - (0, \corner)$);
               \coordinate (se) at ($(ne0) + (0, -\h)$);
               \filldraw [-, line width = \lwidth, fill=white] (nw) -- (ne1) -- (ne2)
                [rounded corners=\cornerradius]--(se) -- (nw|-se) -- cycle;
               \draw [-, line width = \lwidth] (ne1) [rounded corners=\cornerradius]-- (ne1|-ne2) -- (ne2);
               \node [anchor=north west] at (nw) {\fontsize{10}{7}\selectfont \filename};
               \node [anchor=north west, text width=4.8cm] (poem) at ([yshift=-0.6cm]nw) {%
                  \begin{varwidth}{4.8cm}
                      \fontsize{6}{7.2}\selectfont
                      Fliege kleiner Schmetterling \newline
                      mach dir die Welt zu Eigen \newline
                      du kannst mit deiner Leichtigkeit \newline
                      uns noch so vieles zeigen \newline
                      zum Beispiel, dass es weiter geht \newline
                      auch wenn man lang im Dunkel steht \newline
                      am Lebensanfang nur gekrochen \newline
                      hast du die Freiheit nun gerochen \newline
                      du zeigst den Wandel uns im Leben \newline
                      so ist das Leben, Wandel eben. \newline
                      \vspace{\fill}
                  \end{varwidth}
               };
             }

             \draw[-latex, line width=0.7mm] (5.1, -1.7) -- node[above] {Dall-E 2} (7.5, -1.7);

             \node[] (img) at (10, -1.75) {
                \includegraphics[width=4.5cm]{resources/images/DALL·E 2023-04-19.png}
             };

             %\draw[-latex, line width=0.7mm] ([xshift=5mm]poem.east) -- node[above] {Dall-E 2} ([xshift=-5mm]img.west);

        \end{tikzpicture}
    }
    \caption[Beispiel für ein generiertes Vorschaubild]{Ein von Dall-E 2 generiertes Vorschaubild.}
    \label{sec1:intro:subsec:problems:fig:mmir-example-1}
\end{figure}
