\begin{figure}[htb]
    \centering
    \newlength{\w}
    \setlength{\w}{0.9\textwidth}
    \resizebox*{\w}{!}{
        \begin{tikzpicture}

            \tikzset{
                exstyle/.style={-{Triangle[angle=90:6pt,length=3mm,fill=black]}},
                pbstyle/.style={-{Triangle[angle=90:3pt,length=1.5mm,fill=black]}}
            }

            \node[rectangle, minimum width=\w, minimum height=\w-2cm] (outline) {};

            \newlength{\nuna}
            \setlength{\nuna}{0.6\w}

            \node[anchor=south east, draw, rectangle, rounded corners, minimum width=\nuna, minimum height=\nuna] (nuna) at (outline.south east) {};

            \node [anchor=south east, draw, rectangle, rounded corners, text=white, fill=black] at (nuna.south east) {
                \scalebox{0.5}{
                    \textbf{nach J. Nunamaker}   
                }
            };

            \node[rectangle, rounded corners, minimum width=\nuna, minimum height=\nuna, inner sep=0mm] at ([yshift=-0.1cm]nuna.center) {
                \resizebox*{\nuna}{\nuna}{
                    \begin{tikzpicture}[transform shape]

                        \newlength{\radius}
                        \setlength{\radius}{\textwidth-2.5cm}
                        \node[circle, minimum size=\radius] (c) at (nuna.center) {};

                        \def\labellist{
                            \shortstack{
                                \Large \textbf{Beobachtung} \\
                                \rule{3cm}{1pt} \\
                                \footnotesize
                                Umfragestudien, \\ 
                                \footnotesize
                                Fallstudien, \\
                                \footnotesize
                                Feldforschung
                            }, 
                            \shortstack{
                                \Large \textbf{Experiment} \\
                                \rule{3cm}{1pt} \\
                                \footnotesize
                                Computersimulationen,\\ 
                                \footnotesize
                                Feldexperimente, \\
                                \footnotesize
                                Laborexperimente
                            },
                            \shortstack{
                                \Large \textbf{Theorie} \\ \Large \textbf{Bildung} \\
                                \rule{3cm}{1pt} \\
                                \footnotesize
                                konzept. Rahmen-\\ 
                                \footnotesize
                                bedingungen, mathe\\
                                \footnotesize
                                matische Modelle\\
                                \footnotesize 
                                Methoden
                            }
                        }

                        \foreach [count=\i] \x in \labellist {
                            \node (n\i) [draw, fill=white, minimum size=0cm, inner sep=0mm, circle] at (c.\i*360/3+90) {
                                \scalebox{0.8}{
                                    \x
                                }
                            };
                        }
                        \node (nm) [draw, fill=white, minimum size=0cm, inner sep=0mm, circle] at (c.center) {
                            \scalebox{0.8}{
                                \shortstack{
                                    \Large \textbf{System} \\ \Large \textbf{Entwicklung} \\
                                    \rule{3cm}{1pt} \\
                                    \footnotesize
                                    Produktentwicklung,\\ 
                                    \footnotesize
                                    Technologietransfer, \\
                                    \footnotesize
                                    Prototyping
                                }
                            }
                        };

                        \draw[exstyle] (nm.200) -- (n1.40);
                        \draw[exstyle] (n1.25) -- (nm.215);
    
                        \draw[exstyle] (nm.340) -- (n2.140);
                        \draw[exstyle] (n2.155) -- (nm.325);
    
                        \draw[exstyle] (nm.{83}) -- (n3.{277});
                        \draw[exstyle] (n3.{263}) -- (nm.{97});
    
                        \draw[exstyle, bend left=41.5](n3.0) to (n2.60);
                        \draw[exstyle, bend left=41.5](n2.240) to (n1.300);
                        \draw[exstyle, bend left=41.5](n1.120) to (n3.180);
    
                        \draw[exstyle, bend right=41.5](n2.78) to (n3.342);
                        \draw[exstyle, bend right=41.5](n1.318) to (n2.222);
                        \draw[exstyle, bend right=41.5](n3.198) to (n1.102);
                        
                    \end{tikzpicture}
                }
            };

            %\node[draw, rectangle, rounded corners, fill=gray!10] at ([xshift=1.9cm, yshift=1cm]n1.340) {
            %    \tiny Kapitel 2
            %};

            \newlength{\struc}
            \setlength{\struc}{0.47\w}

            \node[anchor=south west, draw, rectangle, rounded corners, minimum width=\struc, minimum height=3.74cm] (struc) at ([yshift=0.4cm]nuna.north west) {};

            \node[anchor=north, minimum width=\struc, draw, rectangle, rounded corners, text=white, fill=black] at (struc.north){
                \scalebox{0.5}{
                    \textbf{Struktur der Arbeit}
                }
            };

            \node[anchor=north west, draw, rectangle, rounded corners, fill=gray!10] (chap1) at ([xshift=0.1cm, yshift=-0.5cm]struc.north west) {
                \tiny Kapitel 1
            };

            \node[anchor=north west, draw, rectangle, rounded corners, fill=gray!10] (chap2) at ([xshift=0.1cm, yshift=-1.04cm]struc.north west) {
                \tiny Kapitel 2
            };

            \node[anchor=north west, draw, rectangle, rounded corners, fill=gray!10] (chap3) at ([xshift=0.1cm, yshift=-1.58cm]struc.north west) {
                \tiny Kapitel 3
            };

            \node[anchor=north west, draw, rectangle, rounded corners, fill=gray!10] (chap4) at ([xshift=0.1cm, yshift=-2.12cm]struc.north west) {
                \tiny Kapitel 4
            };

            \node[anchor=north west, draw, rectangle, rounded corners, fill=gray!10] (chap5) at ([xshift=0.1cm, yshift=-2.66cm]struc.north west) {
                \tiny Kapitel 5
            };

            \node[anchor=north west, draw, rectangle, rounded corners, fill=white] (chap6) at ([xshift=0.1cm, yshift=-3.2cm]struc.north west) {
                \tiny Kapitel 6
            };

            \node[anchor=west] at ([xshift=0.1cm]chap1.east) {
                \tiny Einleitung / Motivation
            };

            \node[anchor=west] at ([xshift=0.1cm]chap2.east) {
                \tiny Stand der Wissenschaft und Technik
            };

            \node[anchor=west] at ([xshift=0.1cm]chap3.east) {
                \tiny Modellierung / Design
            };

            \node[anchor=west] at ([xshift=0.1cm]chap4.east) {
                \tiny Implementierung
            };

            \node[anchor=west] at ([xshift=0.1cm]chap5.east) {
                \tiny Evaluierung
            };

            \node[anchor=west] at ([xshift=0.1cm]chap6.east) {
                \tiny Zusammenfassung / Fazit
            };

            \newlength{\chap}
            \setlength{\chap}{4.2cm}

            \node[anchor=south west, minimum width=\chap, minimum height=8cm, draw, rectangle, rounded corners] (chapter1) at (outline.south west) {};


            
            \node[anchor=center, draw, rectangle, rounded corners, fill=gray!10] (pbn) at ([yshift=-0.8cm]chapter1.north) {
                \tiny PB
            };

            \node[anchor=center, draw, rectangle, rounded corners, fill=gray!10] (pb1n) at ([xshift=-1cm, yshift=-2.8cm]chapter1.north) {
                \tiny PB1
            };

            \node[anchor=center, draw, rectangle, rounded corners, fill=gray!10] (pb2n) at ([xshift=1cm, yshift=-2.8cm]chapter1.north) {
                \tiny PB2
            };

            \node[anchor=center, draw, rectangle, rounded corners, fill=gray!10] (resq1) at ([xshift=-1cm, yshift=-4.5cm]chapter1.north) {
                \tiny FF1
            };

            \node[anchor=center, draw, rectangle, rounded corners, fill=gray!10] (resq2) at ([xshift=1cm, yshift=-4.5cm]chapter1.north) {
                \tiny FF2
            };

            

            \node[anchor=west, draw, rectangle, rounded corners, fill=gray!10] (resto) at ([xshift=-0.4cm, yshift=-5.85cm]chapter1.north) {
                \tiny FZ 1.1/O
            };

            \node[anchor=west, draw, rectangle, rounded corners, fill=gray!10] (resttb) at ([xshift=-0.4cm, yshift=-6.45cm]chapter1.north) {
                \tiny FZ 1.2/TB
            };

            \node[anchor=west, draw, rectangle, rounded corners, fill=gray!10] (resti) at ([xshift=-0.4cm, yshift=-7.05cm]chapter1.north) {
                \tiny FZ 1.3/I
            };

            \node[anchor=west, draw, rectangle, rounded corners, fill=gray!10] (reste) at ([xshift=-0.4cm, yshift=-7.65cm]chapter1.north) {
                \tiny FZ 1.4/E
            };

            \draw[pbstyle, line width=0.3mm] (pbn) -- (pb1n);
            \draw[pbstyle, line width=0.3mm] (pbn) -- (pb2n);

            \draw[pbstyle, line width=0.3mm] (pb1n) -- (resq1);
            \draw[pbstyle, line width=0.3mm] (pb2n) -- (resq2);
            
            \draw[pbstyle, line width=0.3mm] (resq1) |- (reste);
            \draw[pbstyle, line width=0.3mm] (resq1) |- (resti);
            \draw[pbstyle, line width=0.3mm] (resq1) |- (resttb);
            \draw[pbstyle, line width=0.3mm] (resq1) |- (resto);

            

            \node[align=left, anchor=north, minimum width=\chap, draw, rectangle, rounded corners, text=white, fill=black, text width=0.9\chap] (pb) at (chapter1.north){
                \tiny \textbf{Problem}
            };

            \node[align=left, anchor=north, minimum width=\chap, draw, rectangle, rounded corners, text=white, fill=black, text width=0.9\chap] (pbs) at ([yshift=-1.5cm]chapter1.north){
                \tiny \textbf{Problembeschreibungen}
            };

            \node[align=left, anchor=north, minimum width=\chap, draw, rectangle, rounded corners, text=white, fill=black, text width=0.9\chap] (resq) at ([yshift=-3.5cm]chapter1.north){
                \tiny \textbf{Forschungsfragen}
            };

            \node[align=left, anchor=north, minimum width=\chap, draw, rectangle, rounded corners, text=white, fill=black, text width=0.9\chap] (rest) at ([yshift=-5.04cm]chapter1.north){
                \tiny \textbf{Forschungsziele}
            };

            \node[anchor=center, draw, rectangle, rounded corners, fill=gray!10] at (pb.east) {
                \tiny Kap. 1
            };

            \node[anchor=center, draw, rectangle, rounded corners, fill=gray!10] at (pbs.east) {
                \tiny Kap. 1
            };

            \node[anchor=center, draw, rectangle, rounded corners, fill=gray!10] at (resq.east) {
                \tiny Kap. 1
            };

            \draw[exstyle, line width=0.6mm] (chap1.west) -| (chapter1.north);

            \draw[exstyle, line width=0.6mm] ([yshift=1cm]chapter1.south east) -- ([yshift=1cm]nuna.south west);



            \node[draw, rectangle, rounded corners, fill=gray!10] at ([xshift=-3.5cm, yshift=-1.2cm]nuna.center) {
                \tiny Kapitel 2
            };

            \node[draw, rectangle, rounded corners, fill=gray!10] at ([xshift=1.45cm, yshift=3.1cm]nuna.center) {
                \tiny Kapitel 3
            };

            \node[draw, rectangle, rounded corners, fill=gray!10] at ([xshift=1.3cm, yshift=0.2cm]nuna.center) {
                \tiny Kapitel 4
            };

            \node[draw, rectangle, rounded corners, fill=gray!10] at ([xshift=1.6cm, yshift=-2.3cm]nuna.center) {
                \tiny Kapitel 5
            };
            

        \end{tikzpicture}
    }
    \caption{Aufbau der Arbeit und Ansatz der Problemlösung nach Nunamaker \cite{nunamaker}.}
    \label{sec1:intro:subsec:approach-structure:fig:outline-nunamaker}
\end{figure}