\begin{tcolorbox}
    [enhanced,
     colback=white,
     colbacktitle=white,
     coltitle=black,
     title=OpenAI-Java,
     toptitle=1.5mm,
     bottomtitle=1.5mm,
     center title,
     breakable,
     segmentation style={solid}
]
\begin{lstlisting}[
    firstnumber=1,
    language=XML,
    basicstyle=\ttfamily\footnotesize,
    breaklines=true
]
<!-- https://mvnrepository.com/artifact/com.theokanning.openai-gpt3-java/service -->
<dependency>
  <groupId>com.theokanning.openai-gpt3-java</groupId>
  <artifactId>service</artifactId>
  <version>0.13.0</version>
</dependency>
\end{lstlisting}
\tcbline
\textbf{Verwendungszweck}: Diese Abhängigkeit stellt die Schnittstelle von OpenAI und ihre Endpunkte in Java bereit.
\end{tcolorbox}

\begin{tcolorbox}
    [enhanced,
     colback=white,
     colbacktitle=white,
     coltitle=black,
     title=JTokkit,
     toptitle=1.5mm,
     bottomtitle=1.5mm,
     center title,
     breakable,
     segmentation style={solid}
]
\begin{lstlisting}[
    firstnumber=1,
    language=XML,
    basicstyle=\ttfamily\footnotesize,
    breaklines=true
]
<!-- https://mvnrepository.com/artifact/com.knuddels/jtokkit -->
<dependency>
  <groupId>com.knuddels</groupId>
  <artifactId>jtokkit</artifactId>
  <version>0.6.1</version>
</dependency>
\end{lstlisting}
\tcbline
\textbf{Verwendungszweck}: Diese Abhängigkeit bietet Funktionen zum Verarbeiten der von OpenAI genutzten Zeichenaufteilung in Tokens.
\end{tcolorbox}

\begin{tcolorbox}
    [enhanced,
     colback=white,
     colbacktitle=white,
     coltitle=black,
     title=Abhängigkeiten für die Benutzeroberfläche,
     toptitle=1.5mm,
     bottomtitle=1.5mm,
     center title,
     breakable,
     segmentation style={solid}
]
\begin{lstlisting}[
    firstnumber=1,
    language=XML,
    basicstyle=\ttfamily\footnotesize,
    breaklines=true
]
<!-- https://mvnrepository.com/artifact/org.swinglabs/jxlayer -->
<dependency>
  <groupId>org.swinglabs</groupId>
  <artifactId>jxlayer</artifactId>
  <version>3.0.4</version>
</dependency>

<!-- https://mvnrepository.com/artifact/org.swinglabs.swingx/swingx-all -->
<dependency>
  <groupId>org.swinglabs.swingx</groupId>
  <artifactId>swingx-all</artifactId>
  <version>1.6.5</version>
</dependency>

<!-- https://mvnrepository.com/artifact/com.miglayout/miglayout -->
<dependency>
  <groupId>com.miglayout</groupId>
  <artifactId>miglayout</artifactId>
  <version>3.7.4</version>
</dependency>

<!-- https://mvnrepository.com/artifact/com.fifesoft/rsyntaxtextarea -->
<dependency>
  <groupId>com.fifesoft</groupId>
  <artifactId>rsyntaxtextarea</artifactId>
  <version>3.3.1</version>
</dependency>
\end{lstlisting}
\end{tcolorbox}

\begin{tcolorbox}
    [enhanced,
     colback=white,
     colbacktitle=white,
     coltitle=black,
     title=Google Guava,
     toptitle=1.5mm,
     bottomtitle=1.5mm,
     center title,
     breakable,
     segmentation style={solid}
]
\begin{lstlisting}[
    firstnumber=1,
    language=XML,
    basicstyle=\ttfamily\footnotesize,
    breaklines=true
]
<!-- https://mvnrepository.com/artifact/com.google.guava/guava -->
<dependency>
  <groupId>com.google.guava</groupId>
  <artifactId>guava</artifactId>
  <version>32.1.2-jre</version>
</dependency>
\end{lstlisting}
\tcbline
\textbf{Verwendungszweck}: Diese Abhängigkeit bietet Funktionen zur Datenverarbeitung.
\end{tcolorbox}
