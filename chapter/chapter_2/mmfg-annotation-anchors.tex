\begin{figure}[htb]
    \centering
    \resizebox{\textwidth}{!}{
        \begin{tikzpicture}[font=\fontsize{9}{0pt}\selectfont]

            \definecolor{e4draw}{RGB}{224,155,254}
            \definecolor{n5fill}{RGB}{84,224,163}
            \definecolor{node}{RGB}{51, 175, 255}
            \definecolor{compr}{RGB}{224, 155, 254}
            \definecolor{sr}{RGB}{84, 224, 163}

            \tikzset{
                e1/.style={-{Triangle[angle=90:3pt,length=1.5mm,fill=black]}},
                e2/.style={-{Triangle[angle=90:3pt,length=1.5mm,fill=e4draw]}},
                e3/.style={-{Triangle[angle=90:3pt,length=1.5mm,fill=n5fill]}},
                comprstyle/.style={-{Triangle[angle=90:3pt,length=1.5mm,fill=compr]}},
                srstyle/.style={-{Triangle[angle=90:3pt,length=1.5mm,fill=sr]}},
                thick/.style={line width=1.1pt},
                every node/.append style={thick}
            }

            \node[text width=2.5cm] (formal-model) at (0,0) {
                \shortstack{Formale \\Modellierung \\der Darstellung \\eines MMFG}
            };

            \node[anchor=south west, text width=2.5cm] (anchor-annot) at (formal-model.north west) {
                \shortstack{Anker von \\Annotierungen}
            };

            \node[anchor=south west, text width=2.5cm] (init-label) at (anchor-annot.north west) {
                \shortstack{Initiale textuelle \\Annotierung}
            };

            \tikzmath{
                coordinate \p;
                \p = (formal-model.south west) - (init-label.north west);
                \len = veclen(\p);
            }

            \node[anchor=south east, minimum height=\len] (mmfg) at (formal-model.south west) {
                \hspace*{1mm}
                \begin{turn}{90}
                    \fontsize{12}{0}\fontseries{bx}\selectfont
                    MMFG
                \end{turn}
                \hspace*{1mm}
            };

            \tikzmath{
                coordinate \mmfg;
                \mmfg = (mmfg.south west) - (mmfg.south east);
                \mmfglen = veclen(\mmfg);
            }

            \node[draw, rectangle, fit={(mmfg)($(mmfg.south east)+(\textwidth,0)$)}, style={inner sep=0pt}] (f) {};

            \tikzmath{
                coordinate \f;
                \f = (f.south east) - (formal-model.south east);
                \flen = veclen(\f) * 0.0352778;
            }

            \draw[line width=1.1pt] (mmfg.north east) -- (mmfg.south east);
            \draw[] (init-label.north east) -- (formal-model.south east);
            \draw[] (formal-model.north west) -- (formal-model.north east);
            \draw[] (anchor-annot.north west) -- (anchor-annot.north east);

            \path[name path=f-eastl] (f.north east) -- (f.south east);
            \path[name path=div1] (formal-model.north east) -- ++(\flen,0);
            \path[name intersections={of=div1 and f-eastl, by=div-f}];

            \draw[] (formal-model.north east) -- (div-f);

            \path[name path=div2] (anchor-annot.north east) -- ++(\flen,0);
            \path[name intersections={of=div2 and f-eastl, by=div2-f}];

            \draw[] (anchor-annot.north east) -- (div2-f);


            \node[rectangle, draw, fill=node, right=1.5cm of formal-model.east, text width=2cm, align=center] (fvt1) {$S_{MMFG}$};
            \node[rectangle, draw, fill=node, right=1cm of fvt1, text width=2cm, align=center] (fvt2) {$fvt$};
            \node[rectangle, draw, fill=node, right=1cm of fvt2, text width=2cm, align=center] (fvt3) {$fvt$};

            \draw[e1] (fvt1.east) -- node[below] {$frt$} (fvt2.west);
            \draw[e1] (fvt2.east) -- node[below] {$frt$} (fvt3.west);

            \foreach \x in {0,...,4} {
                \node[rectangle, draw, right=1cm + 2.33cm * \x of anchor-annot.east, line width=0.4pt] (aa-\x) {$aa$};
                \node[rectangle, draw, above=0.5cm of aa-\x.north, line width=0.4pt] (lbl-\x) {$LBL$};
            }

            \draw[e1] (fvt1.north) -- (aa-0.south);
            \draw[e1] (fvt1.east) ++(5mm,0cm) -- (aa-1.south);
            \draw[e1] (fvt2.north) -- (aa-2.south);
            \draw[e1] (fvt2.east) ++(5mm,0cm) -- (aa-3.south);
            \draw[e1] (fvt3.north) -- (aa-4.south);

            \draw[e1] (aa-0.north) -- (lbl-0.south);
            \draw[e1] (aa-1.north) -- (lbl-1.south);
            \draw[e1] (aa-2.north) -- (lbl-2.south);
            \draw[e1] (aa-3.north) -- (lbl-3.south);
            \draw[e1] (aa-4.north) -- (lbl-4.south);

        \end{tikzpicture}
    }
    \caption{Formale Modellierung der Darstellung eines MMFGs.}
    \label{sec2:sota:subsec:fz-explainablity:fig:mmfg-formal-model}
\end{figure}