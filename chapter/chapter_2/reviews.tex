{
    \def\arraystretch{1.1}%
    \begin{xltabular}{\linewidth}{
            @{}
            >{
                \hsize=0.2\linewidth
                \centering\arraybackslash
            }X
            >{
                \hsize=0.27\linewidth
                \raggedright\arraybackslash
            }X
            >{
                \hsize=0.27\linewidth
                \raggedright\arraybackslash
            }X
            >{
                \hsize=0.27\linewidth
                \raggedright\arraybackslash
            }X
            @{}
    }

    % First Header

    \caption{Peer-Review-Verfahren.} 
    \label{sec2:sota:table:peer-reviews} \\
        
    \toprule

    \textbf{Verfahren} & \textbf{Verwendung} & \textbf{Ablauf} & \textbf{Vorteile} \\
    
    \midrule
    
    \endfirsthead

    % Normal Head

    \toprule

    \textbf{Verfahren} & \textbf{Verwendung} & \textbf{Ablauf} & \textbf{Vorteile} \\
    
    \midrule
    
    \endhead
        
    % Lower Rows

    Informal Review 
    & Der informale Review ist ein einfaches und flexibles Verfahren, um in erster Instanz nach Defekten zu suchen und Rückmeldungen zu Software-Artifakten zu erhalten, ohne dabei Code auszuführen. 
    In informellen Reviews wird keine Dokumentation angefertigt.
    & 
    & \\
        
    \bottomrule 

    \end{xltabular}
}