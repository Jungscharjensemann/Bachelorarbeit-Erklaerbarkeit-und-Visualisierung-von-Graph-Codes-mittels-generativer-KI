\begin{xltabular}{\linewidth}{
            @{}
            >{
                \hsize=0.2\linewidth
                \centering\arraybackslash
            }X
            >{
                \hsize=0.6\linewidth
            }X
            >{
                \hsize=0.2\linewidth
                \raggedright\arraybackslash
            }X
            @{}
        }

        \caption{Offene Herausforderungen}
        \label{sec2:sota:subsec:summary:table:open-issues}
        \\
        
        \toprule
        % First Row
        \multicolumn{1}{
            >{
                \hsize=0.2\linewidth\centering\arraybackslash
            }X
        }{
            \textbf{Nummer der offenen Herausforderungen}
        } & \multicolumn{1}{
            >{
                \hsize=0.6\linewidth\centering\arraybackslash
            }X
        }{
            \textbf{Inhalt der offenen Herausforderungen}
        } & \multicolumn{1}{
            >{
                \hsize=0.2\linewidth\centering\arraybackslash
            }X
        }{
            \textbf{Ansatz für Lösung}
        } \\

        \midrule

        \endfirsthead

        \toprule
        % First Row
        \multicolumn{1}{
            >{
                \hsize=0.2\linewidth\centering\arraybackslash
            }X
        }{
            \textbf{Nummer der offenen Herausforderungen}
        } & \multicolumn{1}{
            >{
                \hsize=0.6\linewidth\centering\arraybackslash
            }X
        }{
            \textbf{Inhalt der offenen Herausforderungen}
        } & \multicolumn{1}{
            >{
                \hsize=0.2\linewidth\centering\arraybackslash
            }X
        }{
            \textbf{Ansatz für Lösung}
        } \\

        \midrule

        \endhead
        
        % Lower Rows

        \openissue{sec2:sota:oi:1.1}{OH 1.1}
        &
        Offen bleibt, wie durch den Einsatz von generativer KI Erklärbarkeit erreicht werden kann.
        &
        Modellierung, Implementierung
        \\

        \midrule

        \openissue{sec2:sota:oi:1.2}{OH 1.2}
        &
        Offen bleibt, wie Benutzerschnittstellen beschaffen sein müssen, um geeignete Interaktionsmöglichkeiten mit Systemen generativer KI für Anwender des GMAF zu bieten.
        &
        Modellierung, Implementierung
        \\

        \midrule

        \openissue{sec2:sota:oi:2.1}{OH 2.1}
        &
        Offen bleibt, wie die Darstellungsform von Graph Codes in eine geeignete Eingabeform für Systeme generativer KI überführt werden kann.
        &
        Modellierung, Implementierung
        \\

        \midrule

        \openissue{sec2:sota:oi:2.2}{OH 2.2}
        &
        Offen bleibt, wie die Funktionen der Schnittstellen und Endpunkte der vorgestellten und ausgewählten Systeme generativer KI in das GMAF integriert werden können. 
        &
        Modellierung, Implementierung
        \\
        \bottomrule
\end{xltabular}