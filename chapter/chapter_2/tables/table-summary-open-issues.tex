\begin{xltabular}{\linewidth}{
            @{}
            >{
                \hsize=0.2\linewidth
                \centering\arraybackslash
            }X
            >{
                \hsize=0.6\linewidth
            }X
            >{
                \hsize=0.2\linewidth
                \raggedright\arraybackslash
            }X
            @{}
        }

        \caption{Offene Herausforderungen}
        \label{sec2:sota:subsec:summary:table:open-issues}
        \\

        \toprule
        % First Row
        \multicolumn{1}{
            >{
                \hsize=0.2\linewidth\centering\arraybackslash
            }X
        }{
            \textbf{Nummer der offenen Herausforderungen}
        } & \multicolumn{1}{
            >{
                \hsize=0.6\linewidth\centering\arraybackslash
            }X
        }{
            \textbf{Inhalt der offenen Herausforderungen}
        } & \multicolumn{1}{
            >{
                \hsize=0.2\linewidth\centering\arraybackslash
            }X
        }{
            \textbf{Ansatz für Lösung}
        } \\

        \midrule

        \endfirsthead

        \toprule
        % First Row
        \multicolumn{1}{
            >{
                \hsize=0.2\linewidth\centering\arraybackslash
            }X
        }{
            \textbf{Nummer der offenen Herausforderungen}
        } & \multicolumn{1}{
            >{
                \hsize=0.6\linewidth\centering\arraybackslash
            }X
        }{
            \textbf{Inhalt der offenen Herausforderungen}
        } & \multicolumn{1}{
            >{
                \hsize=0.2\linewidth\centering\arraybackslash
            }X
        }{
            \textbf{Ansatz für Lösung}
        } \\

        \midrule

        \endhead

        % Lower Rows

        \hyperref{sec2:sota:oi:1.1}{OH 1.1}
        &
        Es existiert keine Möglichkeit oder Untersuchung Systeme generativer KI zur Erklärbarkeit von Graph Codes zu nutzen.
        Es bleibt daher offen, ob und wie durch den Einsatz von generativer KI Erklärbarkeit erreicht werden kann.
        % Offen bleibt, wie durch den Einsatz von generativer KI Erklärbarkeit erreicht werden kann.
        &
        Modellierung, Implementierung
        \\

        \midrule

        \hyperref{sec2:sota:oi:1.2}{OH 1.2}
        &
        % Es existiert keine Benutzungsschnittstelle im GMAF, die geeignete Interaktionsmöglichkeiten für Systeme generativer KI zur Bild- oder Textgenerierung ermöglicht.
        Im GMAF ist keine Benutzungsschnittstelle vorhanden, die geeignete Interaktionsmöglichkeiten für Systeme generativer KI zur Bild- oder Textgenerierung ermöglicht.
        Es bleibt daher offen, wie eine Benutzungsschnittstelle beschaffen sein muss, um geeignete Interaktionsmöglichkeiten mit Systemen generativer KI für Anwender des GMAF zu bieten.
        % Offen bleibt, wie Benutzungsschnittstellen beschaffen sein müssen, um geeignete Interaktionsmöglichkeiten mit Systemen generativer KI für Anwender des GMAF zu bieten.
        &
        Modellierung, Implementierung
        \\

        \midrule

        \hyperref{sec2:sota:oi:2.1}{OH 2.1}
        &
        Es existiert kein Algorithmus zum Transformieren der in Graph Codes gespeicherten Informationen in eine passende Eingabeform für Systeme generativer KI.
        Es bleibt daher offen, wie die Darstellungsform von Graph Codes in eine geeignete Eingabeform für Systeme generativer KI überführt werden kann.
        % Offen bleibt, wie die Darstellungsform von Graph Codes in eine geeignete Eingabeform für Systeme generativer KI überführt werden kann.
        &
        Modellierung, Implementierung
        \\

        \midrule

        \hyperref{sec2:sota:oi:2.2}{OH 2.2}
        &
        Es existiert keine Integration der von OpenAI angebotenen Systeme generativer KI im GMAF.
        Es bleibt daher offen, wie die Funktionen der Schnittstellen und Endpunkte der vorgestellten und ausgewählten Systeme generativer KI in das GMAF integriert werden können.
        % Offen bleibt, wie die Funktionen der Schnittstellen und Endpunkte der vorgestellten und ausgewählten Systeme generativer KI in das GMAF integriert werden können.
        &
        Modellierung, Implementierung
        \\
        \bottomrule
\end{xltabular}
