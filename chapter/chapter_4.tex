\section{Implementierung}
\label{sec4:impl}
Dieses Kapitel widmet sich der Implementierung und es werden, entsprechend der Methodik von Nunamaker \cite{nunamaker}, alle Forschungsziele der Implementierungsphase behandelt.
Ziel der Implementierung ist die Untersuchung und Validierung der in \cref{sec3:model} beschriebenen Modellierung, und die Vorbereitung einer detaillierten Evaluierung in der Experimentphase, sowie Untersuchungen weiterer Beobachtungen, die während der Implementierung gemacht wurden.
Die Struktur dieses Kapitels folgt der logischen Abfolge der Forschungsziele der Implementierungsphase.
Doch zuvor werden in \cref{sec4:impl:subsec:basics} die Grundlagen bzw. Vorbereitungen für die Implementierung getroffen.
Dies umfasst die Beschreibung der geplanten Struktur und Architektur der Implementierung, sowie weitere technische Voraussetzungen und die in dieser prototypischen Implementierung genutzten Bibliotheken bzw. Abhängigkeiten.
In \cref{sec4:impl:subsec:fz-explainability} werden ausgewählte Implementierungsdetails der Erklärbarkeit von MMIR mittels generativer KI vorgestellt.
Dies umfasst besonders die Benutzeroberfläche, ihre Benutzerelemente und die Interaktion mit diesen.
Analog dazu werden in \cref{sec4:impl:subsec:fz-integration} ausgewählte Implementierungsdetails der Integration generativer KI in das GMAF vorgestellt.
Dies umfasst die Transformation von Graph Codes und die Integration der Schnittstelle von OpenAI und ihre Endpunkte.

{
    \def\arraystretch{1.1}%
    \begin{xltabular}{\linewidth}{
            @{}
            >{
                \hsize=0.25\linewidth
                \raggedright\arraybackslash
            }X
            >{
                \hsize=0.55\linewidth
                \raggedright\arraybackslash
            }X
            >{
                \hsize=0.2\linewidth
                \centering\arraybackslash
            }X
            @{}
    }

    % First Header

    \caption{Struktur des Kapitels 4 \enquote{Implementierung}.}
    \label{sec4:impl:table:structure} \\

    \toprule

    \textbf{FZ} & \textbf{Kurze Beschreibung} & \textbf{Abschnitt} \\

    \midrule

    \endfirsthead

    % Normal Head

    \toprule

    \textbf{FZ} & \textbf{Kurze Beschreibung} & \textbf{Abschnitt} \\

    \midrule

    \endhead

    % Lower Rows

    Grundlagen der Implementierung & Technische Grundlagen & \cref{sec4:impl:subsec:basics} \\
    FZ 1.3/I & Implementierung: Benutzerschnittstelle & \cref{sec4:impl:subsec:fz-explainability} \\
    FZ 2.3/I & Implementierung: Transformation von Graph Codes \& Integration gen. KI & \cref{sec4:impl:subsec:fz-integration} \\
    Zusammenfassung & Ergebnisse / Erkenntnisse & \cref{sec4:impl:subsec:summary} \\

    \bottomrule

    \end{xltabular}
}


\subsection{Grundlagen der Implementierung}
\label{sec4:impl:subsec:basics}
In diesem Abschnitt werden die Grundlagen für die in diesem Kapitel beschriebene prototypische Proof-of-Concept Implementierung festgehalten.
Dies umfasst die allgemeine Architektur, ihr zugrunde liegendes Prinzip, ihr hierarchischer Aufbau, sowie die Abhängigkeiten zu Drittanbieter-Bibliotheken.

\subsubsection{Architekturmuster}
Die Implementierung der Software folgt dem Muster des \textit{Model-View-Controller}-Prinzips und teilt die Architektur der Software in drei separate Bereiche: \textit{Model} für die Datenstrukturen, \textit{View} für das Frontend bzw. die Benutzungsschnittstellen und \textit{Controller} zum Handhaben von Ereignissen und Interaktionen zwischen den Bereichen \textit{Model} und \textit{View}.
Ziel dieser Architektur ist durch das gezielte Trennen dieser Bereiche die Eigenschaften wie Wiederverwendbarkeit, Wartbarkeit und Robustheit der Software zu erhöhen und zu stärken.

Das erste Forschungsziel \hyperref[sec4:impl:subsec:fz-explainability]{FZ 1.3/I} widmet sich der Benutzungsschnittstelle, ihren Elementen und der Interaktion mit diesen und behandelt somit die Bereiche \textit{View} und \textit{Controller}.
Das zweite Forschungsziel \hyperref[sec4:impl:subsec:fz-integration]{FZ 2.3/I} widmet sich der Transformation von Graph Codes, sowie der Integration der Schnittstelle von OpenAI, ihrer Endpunkte und dem Erstellen von Anfragen an diese Endpunkte und behandelt somit die Bereiche \textit{Controller} und \textit{Model}.

\cref{sec4:impl:subsec:basiscs:fig:hierarchy} zeigt die Hierarchie des Quellcodes, der in diesem Kapitel beschriebenen prototypischen Proof-of-Concept Implementierung.

\def\Size{4pt}
\tikzset{
      folder/.pic={
        \filldraw[draw=folderborder,top color=folderbg!50,bottom color=folderbg]
          (-1.05*\Size,0.2\Size+5pt) rectangle ++(.75*\Size,-0.2\Size-5pt);  
        \filldraw[draw=folderborder,top color=folderbg!50,bottom color=folderbg]
          (-1.15*\Size,-\Size) rectangle (1.15*\Size,\Size);
      }
    }

\begin{figure}[htb]
\begin{forest}
      for tree={
        font=\ttfamily,
        grow'=0,
        child anchor=west,
        parent anchor=south,
        anchor=west,
        calign=first,
        inner xsep=7pt,
        edge path={
          \noexpand\path [draw, \forestoption{edge}]
          (!u.south west) +(7.5pt,0) |- (.child anchor) pic {folder} \forestoption{edge label};
        },
        % style for your file node 
        file/.style={edge path={\noexpand\path [draw, \forestoption{edge}]
          (!u.south west) +(7.5pt,0) |- (.child anchor) \forestoption{edge label};},
          inner xsep=2pt,font=\small\ttfamily
                     },
        before typesetting nodes={
          if n=1
            {insert before={[,phantom]}}
            {}
        },
        fit=band,
        before computing xy={l=15pt},
      }  
    [de
        [ja
            [controller]
            [model]
            [view]
        ]
        [swa]
    ]
\end{forest}
\caption{Hierarchie des Quellcodes.}
\label{sec4:impl:subsec:basiscs:fig:hierarchy}
\end{figure}

\subsubsection{Abhängigkeiten (Maven)}
Für die prototypische Proof-of-Concept Implementierung der in \cref{sec3:model} vorgestellten konzeptuellen Modellierung wird die Programmiersprache Java \cite{java} ausgewählt.
Der in diesem Kapitel vorgestellte Quellcode ist auf GitHub verfügbar.
Die Implementierung in Java baut hierbei auf Maven \cite{maven}, einem Build-Tool der Apache Software Foundation für die Projektverwaltung und Management von Abhängigkeiten, auf.
Im Folgenden werden die in diesem Projekt verwendet Abhängigkeiten aufgezählt und ihre Verwendungszwecke genannt.

\begin{tcolorbox}
    [enhanced,
     colback=white,
     colbacktitle=white,
     coltitle=black,
     title=OpenAI-Java,
     toptitle=1.5mm,
     bottomtitle=1.5mm,
     center title,
     breakable,
     segmentation style={solid}
]
\begin{lstlisting}[
    firstnumber=1,
    language=XML,
    basicstyle=\ttfamily\footnotesize,
    breaklines=true
]
<!-- https://mvnrepository.com/artifact/com.theokanning.openai-gpt3-java/service -->
<dependency>
  <groupId>com.theokanning.openai-gpt3-java</groupId>
  <artifactId>service</artifactId>
  <version>0.13.0</version>
</dependency>
\end{lstlisting}
\tcbline
\textbf{Verwendungszweck}: Diese Abhängigkeit stellt die Schnittstelle von OpenAI und ihre Endpunkte in Java bereit.
\end{tcolorbox}

\begin{tcolorbox}
    [enhanced,
     colback=white,
     colbacktitle=white,
     coltitle=black,
     title=JTokkit,
     toptitle=1.5mm,
     bottomtitle=1.5mm,
     center title,
     breakable,
     segmentation style={solid}
]
\begin{lstlisting}[
    firstnumber=1,
    language=XML,
    basicstyle=\ttfamily\footnotesize,
    breaklines=true
]
<!-- https://mvnrepository.com/artifact/com.knuddels/jtokkit -->
<dependency>
  <groupId>com.knuddels</groupId>
  <artifactId>jtokkit</artifactId>
  <version>0.6.1</version>
</dependency>
\end{lstlisting}
\tcbline
\textbf{Verwendungszweck}: Diese Abhängigkeit bietet Funktionen zum Verarbeiten der von OpenAI genutzten Zeichenaufteilung in Tokens.
\end{tcolorbox}

\begin{tcolorbox}
    [enhanced,
     colback=white,
     colbacktitle=white,
     coltitle=black,
     title=Abhängigkeiten für die Benutzeroberfläche,
     toptitle=1.5mm,
     bottomtitle=1.5mm,
     center title,
     breakable,
     segmentation style={solid}
]
\begin{lstlisting}[
    firstnumber=1,
    language=XML,
    basicstyle=\ttfamily\footnotesize,
    breaklines=true
]
<!-- https://mvnrepository.com/artifact/org.swinglabs/jxlayer -->
<dependency>
  <groupId>org.swinglabs</groupId>
  <artifactId>jxlayer</artifactId>
  <version>3.0.4</version>
</dependency>

<!-- https://mvnrepository.com/artifact/org.swinglabs.swingx/swingx-all -->
<dependency>
  <groupId>org.swinglabs.swingx</groupId>
  <artifactId>swingx-all</artifactId>
  <version>1.6.5</version>
</dependency>

<!-- https://mvnrepository.com/artifact/com.miglayout/miglayout -->
<dependency>
  <groupId>com.miglayout</groupId>
  <artifactId>miglayout</artifactId>
  <version>3.7.4</version>
</dependency>

<!-- https://mvnrepository.com/artifact/com.fifesoft/rsyntaxtextarea -->
<dependency>
  <groupId>com.fifesoft</groupId>
  <artifactId>rsyntaxtextarea</artifactId>
  <version>3.3.1</version>
</dependency>
\end{lstlisting}
\end{tcolorbox}

\begin{tcolorbox}
    [enhanced,
     colback=white,
     colbacktitle=white,
     coltitle=black,
     title=Google Guava,
     toptitle=1.5mm,
     bottomtitle=1.5mm,
     center title,
     breakable,
     segmentation style={solid}
]
\begin{lstlisting}[
    firstnumber=1,
    language=XML,
    basicstyle=\ttfamily\footnotesize,
    breaklines=true
]
<!-- https://mvnrepository.com/artifact/com.google.guava/guava -->
<dependency>
  <groupId>com.google.guava</groupId>
  <artifactId>guava</artifactId>
  <version>32.1.2-jre</version>
</dependency>
\end{lstlisting}
\tcbline
\textbf{Verwendungszweck}: Diese Abhängigkeit bietet Funktionen zur Datenverarbeitung.
\end{tcolorbox}


\clearpage

\subsection[FZ 1.3/I Erklärbarkeit von MMIR mittels generativer KI]{\texorpdfstring{FZ 1.3/I Erklärbarkeit von MMIR mittels \\ generativer KI}{FZ 1.3/I Erklärbarkeit von MMIR mittels generativer KI}}
\label{sec4:impl:subsec:fz-explainability}

In diesem Abschnitt wird die prototypische Proof-of-Concept Implementierung der \enquote{Erklärbarkeit von MMIR mittels generativer KI} beschrieben und behandelt.
Die in \cref{sec3:model} identifizierten Anwendungsfälle beschreiben die Interaktionen eines Benutzers mit dem System.
Um eine geschickte Interaktion mit dem System zu ermöglichen, wird eine Benutzerschnittstelle mit adäquaten Interaktionsmöglichkeiten benötigt.
Die Beschreibung der Implementierung solch einer Benutzerschnittstelle wird im Folgenden in \cref{sec4:impl:subsubsec:ui} vorgenommen.
Diese Beschreibung umfasst die Elemente der Benutzeroberfläche, sowie die Interaktion zwischen diesen.

\subsubsection{Benutzerschnittstelle}
\label{sec4:impl:subsubsec:ui}
In diesem Abschnitt wird die Implementierung der Benutzeroberfläche beschrieben.
Für eine prototypische Proof-of-Concept Implementierung der Benutzeroberfläche wird Java's \textit{Swing} verwendet.
\textit{Swing} ist ein Werkzeug für das Erstellen von grafischen Benutzerschnittstellen bzw. oberflächen und ist ein grundlegender Bestandteil der Laufzeitumgebung von Java.
Das Benutzen von \textit{Swing} hat somit den Vorteil, dass keine zusätzliche Installation notwendig ist und \textit{Swing} somit problemlos auf beliebigen Rechnern, auf welchen Java installiert ist, ausgeführt werden kann.
\begin{figure}[!ht]
  \includegraphics[width=8cm]{chapter/chapter_4/wireframe-impl-overview}
  \caption{Allgemeine Übersicht über das Mockup der Benutzerschnittstelle.}
  \label{sec4:impl:subsubsec:ui:fig:wireframe-overview}
\end{figure}

\cref{sec4:impl:subsubsec:ui:fig:wireframe-overview} zeigt eine allgemeine Übersicht über das bereits in \cref{sec3:model:par:wireframe:fig:stage-1} vorgestellte Mockup der Benutzerschnittstelle, erweitert um Zuweisungen der Komponenten mitsamt implementierungsspezifischen Paketnamen.
Die Elemente der in \cref{sec4:impl:subsubsec:ui:fig:wireframe-overview} gezeigten Benutzerschnittstelle werden in \cref{sec4:impl:par:ui-elements} genauer beschrieben.
Aufbauend auf diesen Elementen wird dann in \cref{sec4:impl:par:ui-interaction} die Interaktion zwischen diesen Elementen beschrieben.
Auf diese Weise entspricht der Ablauf dieses Abschnitts dem Ablauf der Modellierung der Benutzerschnittstelle.

\paragraph{Elemente der Benutzeroberfläche}
\label{sec4:impl:par:ui-elements}

Das Fundament der Benutzerschnittstelle ist das \textit{ExplainerFrame}, welches durch die Klasse \textit{ja.view.ExplainerFrame} umgesetzt wird.
\cref{sec4:impl:par:ui-elements:lst:explainer-frame} zeigt die wichtigsten Aspekte der Implementierung dieser Klasse.

\lstinputlisting[style=java-code, caption={ExplainerFrame-Klasse}, label={sec4:impl:par:ui-elements:lst:explainer-frame}]{chapter/chapter_4/java/ExplainerFrame.java}

Nach der Modellierung enthält dieses Fundament drei Grundbereiche mit folgender Nummerierung (siehe \cref{sec3:model:par:wireframe:fig:stage-1}): \circitem{1} EditorGraphCode, \circitem{2} ExplainerPanel und \circitem{3} ExplainerConsole.
Diese Grundbereiche werden in \cref{sec4:impl:par:ui-elements:lst:explainer-frame} durch die privaten Variablen \textit{editorGraphCode}, \textit{explanationPanel} und \textit{explainerConsole} dargestellt.
Beim Erzeugen der Klasse \textit{ExplainerFrame} wird zuerst durch die Methode \textit{initFrame()} das Frame initialisiert und konfiguriert.
Dies umfasst die Dimension des Frames, die Position, sowie den Titel.
Darauffolgend werden dann durch die Methode \textit{initComponents()} die Komponenten, welches auch die Grundbereiche umfassen, initialisiert, konfiguriert und dem Frame hinzugefügt.
Dies umfasst besonders die notwendigen Schritte zum Hinzufügen und Layouten (z.Dt. Auslegen) der Grundbereiche im \textit{ExplainerFrame}.
Im weiteren Verlauf wird zuerst der Grundbereich \textit{EditorGraphCode}, welcher die linke Arbeitsfläche darstellt, dann der Grundbereich \textit{ExplanationPanel}, welcher die rechte Arbeitsfläche darstellt und schlussendlich der Grundbereich \textit{ExplainerConsole}, welcher die Konsole darstellt, behandelt.
Jeder dieser Grundbereiche wird mitsamt seiner enthaltenen Komponenten ausführlich behandelt, bevor mit dem nächsten Grundbereich fortgefahren wird.

\begin{figure}[!ht]
  \includegraphics[width=10cm]{chapter/chapter_4/wireframe-impl-left}
  \caption{Grundbereich \textit{EditorGraphCode}.}
  \label{sec4:impl:subsubsec:ui:fig:wireframe-editor-graph-code}
\end{figure}

\cref{sec4:impl:subsubsec:ui:fig:wireframe-editor-graph-code} zeigt den Grundbereich \textit{EditorGraphCode}, umgesetzt durch die Klasse \textit{ja.view.editor.EditorGraphCode}, in welchem Benutzer durch geeignete Interaktionsmöglichkeiten die Bearbeitung von Graph Code Dateien durchführen können.
\cref{sec4:impl:par:ui-elements:lst:editor-graph-code-p1,sec4:impl:par:ui-elements:lst:editor-graph-code-p2,sec4:impl:par:ui-elements:lst:editor-graph-code-p3,sec4:impl:par:ui-elements:lst:editor-graph-code-p4} zeigen im Folgenden die wichtigsten Aspekte der Implementierung dieser Klasse.

\lstinputlisting[style=java-code, caption={EditorGraphCode-Klasse}, label={sec4:impl:par:ui-elements:lst:editor-graph-code-p1}]{chapter/chapter_4/java/egc/EditorGraphCode-P1.java}

Der Modellierung nach ist diese Arbeitsfläche in zwei nebeneinanderliegende Arbeitsflächen aufgeteilt, die in \cref{sec4:impl:par:ui-elements:lst:editor-graph-code-p1} durch die Variablen \textit{leftPart} und \textit{rightPart} dargestellt werden.
Auch diese Arbeitsflächen bestehen wiederum aus mehreren Komponenten.
Die linke Arbeitsfläche umfasst eine Schnittstelle zum Auswählen und Ausführen von Aktionen (siehe \cref{sec3:model:par:wireframe:fig:stage-2+3} \circitem{4}), dargestellt durch die Variable \textit{operationsPanel} und eine Liste an Graph Code Dateien (siehe \cref{sec3:model:par:wireframe:fig:stage-2+3} \circitem{5}), dargestellt durch die Variable \textit{graphCodeList}.
Die rechte Arbeitsfläche hingegen umfasst im Wesentlichen eine Tabelle zur Visualisierung eines ausgewählten Graph Codes (siehe \cref{sec3:model:par:wireframe:fig:stage-2+3} \circitem{6}), dargestellt durch die Variable \textit{graphCodeTable}, sowie ein Label zum Darstellen des Namens der entsprechenden Graph Code Datei über dieser Tabelle, dargestellt durch die Variable \textit{graphCodeName}.

Im Folgenden wird in \cref{sec4:impl:par:ui-elements:lst:editor-graph-code-p2}, dem zweiten Teil der EditorGraphCode-Klasse, die Schnittstelle \circitem{4} zum Auswählen und Ausführen von Aktionen genauer spezifiziert und im weiteren Verlauf erklärt.

\lstinputlisting[style=java-code, caption={EditorGraphCode-Klasse (Zweiter Teil)}, label={sec4:impl:par:ui-elements:lst:editor-graph-code-p2}, firstnumber=29]{chapter/chapter_4/java/egc/EditorGraphCode-P2.java}

Die Schnittstelle zum Auswählen und Ausführen von Aktionen, dargestellt durch \textit{operationsPanel}, besitzt vier Elemente.
Diese Elemente sind JButtons bzw. Knöpfe und eine JComboBox bzw. ein Button mit einer aus- und einklappbaren Auswahlliste und werden im Folgenden aufgezählt:
\begin{itemize}
  \item Knopf \enquote{Select Graph Code(s)}.
  Dieser Knopf ist mit dem Anwendungsfall \hyperref[sec3:model:uc-1.1]{UC-1.1} \enquote{Graph Code(s) importieren} verbunden und wird im Quellcode als Variable \textit{openGraphCodeChooserButton} eingeführt.
  Weiterhin wird die Interaktion durch ein Steuerelement \textit{ImportGraphCodesController}, welches in einem Mechanismus (siehe \cref{sec3:model:par:mechanism-use-cases:fig:mech-uc-1.1}) identifiziert werden konnte, gesteuert.
  Das an diesem Anwendungsfall beteiligte Steuerelement \textit{ImportGraphCodesController} wird in ... genauer behandelt.
  \item Knopf \enquote{Remove selected Graph Code(s)}.
  Dieser Knopf ist mit dem Anwendungsfall \hyperref[sec3:model:uc-1.2]{UC-1.2} \enquote{Graph Code(s) entfernen} verbunden und wird im Quellcode als Variable \textit{removeSelectedButton} eingeführt.
  Weiterhin wird die Interaktion durch ein Steuerelement \textit{RemoveSelectedGraphCodesController}, welches in einem Mechanismus (siehe \cref{sec3:model:par:mechanism-use-cases:fig:mech-uc-1.2}) identifiziert werden konnte, gesteuert.
  Das an diesem Anwendungsfall beteiligte Steuerelement \textit{RemoveSelectedGraphCodesController} wird in ... genauer behandelt.
  \item Aus- und einklappbare Auswahlliste.
  Diese Auswahlliste ist mit dem Anwendungsfall \hyperref[sec3:model:uc-1.4]{UC-1.4} \enquote{Operation auswählen} verbunden und wird im Quellcode als Variable \textit{calculationComboBox} eingeführt.
  Die Interaktion mit diesem Element wird durch das Steuerelement \textit{GraphCodeCalculationController}, welches in einem Mechanismus (siehe \cref{sec3:model:par:mechanism-use-cases:fig:mech-uc-1.5}) identifiziert werden konnte, gesteuert.
  Das an diesem Anwendungsfall beteiligte Steuerelement \textit{GraphCodeCalculationController} wird in ... genauer behandelt.
  \item Knopf \enquote{Execute}.
  Dieser Knopf ist mit dem Anwendungsfall \hyperref[sec3:model:uc-1.5]{UC-1.5} verbunden und wird im Quellcode als Variable \textit{calculationOperationButton} eingeführt.
  Weiterhin wird die Interaktion ebenfalls durch das Steuerelement \textit{GraphCodeCalculationController} gesteuert.
\end{itemize}
Schlussendlich ist in \cref{sec4:impl:par:ui-elements:lst:editor-graph-code-p2} zu sehen, wie die Elemente der Schnittstelle hinzugefügt werden.
Die aus der Implementierung resultierende Schnittstelle zum Auswählen und Ausführen von Aktionen kann in \cref{sec4:impl:par:ui-elements:fig:wireframe-ui-4} eingesehen werden.

\begin{figure}[!ht]
  \includegraphics[width=9cm]{chapter/chapter_4/wireframe-impl-ui-4}
  \caption{Aus der Implementierung resultierende Schnittstelle zum Auswählen und Ausführen von Aktionen.}
  \label{sec4:impl:par:ui-elements:fig:wireframe-ui-4}
\end{figure}

Im Folgenden wird in \cref{sec4:impl:par:ui-elements:lst:editor-graph-code-p3}, dem dritten Teil der EditorGraphCode-Klasse, die Schnittstelle \circitem{5} bzw. Liste von Graph Codes genauer spezifiziert und im weiteren Verlauf erklärt.

\lstinputlisting[style=java-code, caption={EditorGraphCode-Klasse (Dritter Teil)}, label={sec4:impl:par:ui-elements:lst:editor-graph-code-p3}, firstnumber=52]{chapter/chapter_4/java/egc/EditorGraphCode-P3.java}

Zuerst werden Eigenschaften der Liste konfiguriert.
Eine dieser Eigenschaften ist z.B. der Auswahlmodus \textit{MULTIPLE\_INTERVAL\_SELECTION}, um ein mehrfache Auswahl in der Liste zu ermöglichen.

Die Liste ist mit dem Anwendungsfall \hyperref[sec3:model:uc-1.3]{UC-1.3} \enquote{Graph Code(s) auswählen} verbunden.
In Bezug auf den Anwendungsfall wird die Interaktion mit der Liste durch das Steuerelement \textit{GraphCodeSelectionListener} gesteuert, welches in einem Mechanismus (siehe \cref{sec3:model:par:mechanism-use-cases:fig:mech-uc-1.3}) identifiziert werden konnte.
Das Steuerelement \textit{GraphCodeSelectionListener} wird in ... genauer behandelt.
Zusätzlich besitzt die Liste weitere Steuerelemente, wie \textit{GraphCodeListMouseAdapter} und \textit{ListItemTransferHandler}, die ebenfalls in ... genauer behandelt werden.
Des Weiteren konnte in diesem Mechanismus, wie bereits auch schon in den vorherigen Mechanismen, die Komponente \textit{GraphCodeListElement} identifiziert werden.
Diese Komponenten ist in Bezug auf die Liste von besonderer Bedeutung, da diese ein Element bzw. Eintrag in der Liste darstellt.
Der Quellcode für diese Komponente kann in \cref{sec4:impl:par:ui-elements:lst:gcle} eingesehen werden.

\lstinputlisting[style=java-code, caption={GraphCodeListElement-Klasse}, label={sec4:impl:par:ui-elements:lst:gcle}, firstnumber=1]{chapter/chapter_4/java/gcle/GraphCodeListElement.java}

\cref{sec4:impl:par:ui-elements:fig:wireframe-ui-5} zeigt ein Beispiel für die aus der Implementierung resultierende Liste mitsamt beispielhaften Einträgen für Graph Codes.

\begin{figure}[!ht]
  \includegraphics[width=9cm]{chapter/chapter_4/wireframe-impl-ui-5}
  \caption{Aus der Implementierung resultierende Liste mit beispielhaften Einträgen für Graph Codes.}
  \label{sec4:impl:par:ui-elements:fig:wireframe-ui-5}
\end{figure}

Schlussendlich wird in \cref{sec4:impl:par:ui-elements:lst:gct} die letzte Schnittstelle \circitem{6} \textit{GraphCodeTable} des Grundbereichs \textit{EditorGraphCode} behandelt und im weiteren Verlauf erklärt.

\lstinputlisting[style=java-code, caption={GraphCodeTable-Klasse}, label={sec4:impl:par:ui-elements:lst:gct}, firstnumber=1]{chapter/chapter_4/java/gct/GraphCodeTable.java}

Die Schnittstelle \textit{GraphCodeTable} besitzt zwei Elemente: Eine Tabelle zur Darstellung eines ausgewählten Graph Codes, umgesetzt durch ein \textit{JTable} und ein Platzhalter, zum Signalisieren, dass zu diesem Zeitpunkt noch kein Graph Code ausgewählt ist / wurde.
Die Tabelle wird in einem \textit{JScrollPane} eingebettet, um auch größere Tabellen darstellen zu können.
Des Weiteren ist ein wichtiger Teil dieser Klasse ein Datenmodell für die Tabelle, dargestellt durch die Komponente \textit{GraphCodeTableModel}.
Dieses Datenmodell wird der Tabelle zugewiesen und hat zur Aufgabe, die Informationen in einem Graph Code auf die Zeilen und Spalten der Tabelle abzubilden, sodass diese in Tabellenform dargestellt werden können.
Die wichtigste Methode dieser Klasse ist die Methode \textit{setGraphCode}.
Diese Methode nimmt als Parameter einen ausgewählten Graph Code.
Anhand dieses Graph Codes werden Veränderungen an der Schnittstelle, sowie dem Datenmodell vorgenommen.
Sofern kein Graph Code ausgewählt ist, wird ein Platzhalter angezeigt.
Andernfalls wird dem Datenmodell der Graph Code zur Verarbeitung zugewiesen.

\cref{sec4:impl:par:ui-elements:fig:wireframe-ui-6} zeigt ein Beispiel für die aus der Implementierung resultierende Tabelle zur Darstellung eines ausgewählten Graph Codes.

\begin{figure}[!ht]
  \includegraphics[width=8cm, keepaspectratio]{chapter/chapter_4/wireframe-impl-ui-6}
  \caption{Aus der Implementierung resultierende Tabelle zur Darstellung eines ausgewählten Graph Codes.}
  \label{sec4:impl:par:ui-elements:fig:wireframe-ui-6}
\end{figure}

Im Folgenden wird in \cref{sec4:impl:par:ui-elements:lst:editor-graph-code-p4}, der vierte und letzte Teil der EditorGraphCode-Klasse, genauer spezifiziert und im weiteren Verlauf erklärt.

\lstinputlisting[style=java-code, caption={EditorGraphCode-Klasse (Letzter Teil)}, label={sec4:impl:par:ui-elements:lst:editor-graph-code-p4}, firstnumber=58]{chapter/chapter_4/java/egc/EditorGraphCode-P4.java}

\cref{sec4:impl:par:ui-elements:lst:editor-graph-code-p4} zeigt das abschließende Zusammenfügen der Schnittstellen \circitem{4}, \circitem{5} und \circitem{6}.
Zusammengefügt ergeben diese Schnittstellen den Grundbereich \textit{EditorGraphCode}.
Damit ist der Grundbereich \circitem{1}, \textit{EditorGraphCode}, abgeschlossen.
\cref{sec4:impl:par:ui-elements:fig:wireframe-ui-left-complete} zeigt abschließend die vollständige Benutzeroberfläche des aus der Implementierung resultierenden Grundbereichs \textit{EditorGraphCode} mitsamt Beispielen.

\begin{figure}[!ht]
  \includegraphics[width=\textwidth]{chapter/chapter_4/wireframe-impl-ui-left-complete}
  \caption{Vollständige Oberfläche des aus der Implementierung resultierenden Grundbereichs \textit{EditorGraphCode}.}
  \label{sec4:impl:par:ui-elements:fig:wireframe-ui-left-complete}
\end{figure}

% Überleitung zum nächsten Grundbereich ExplanationPanel...
% Dann mit Grundbereich ExplanationPanel fortfahren...

\clearpage

\begin{figure}[!ht]
  \includegraphics[width=\textwidth]{chapter/chapter_4/wireframe-impl-right}
  \caption{Grundbereich \textit{ExplanationPanel} zur Generierung von visuellen Erklärungen (links) und textuellen Erklärungen (rechts).}
  \label{sec4:impl:subsubsec:ui:fig:wireframe-explanation-panel}
\end{figure}

\cref{sec4:impl:subsubsec:ui:fig:wireframe-explanation-panel} zeigt den Grundbereich \textit{ExplanationPanel}, umgesetzt durch die Klasse \textit{ja.view.explanation.ExplanationPanel}, in welchem Benutzer durch geeignete Interaktionsmöglichkeiten die Generierung von visuellen und textuellen Erklärungen zu Graph Codes durchführen können.
Hierzu besitzt der Grundbereich \textit{ExplanationPanel} eine Fläche, in der Komponenten miteinander ausgewechselt werden können.
Genauer wird zwischen zwei Komponenten gewechselt: \textit{ImagePanel} und \textit{TextPanel}.

Listings aufteilen

\lstinputlisting[style=java-code, caption={ExplanationPanel-Klasse}, label={sec4:impl:par:ui-elements:lst:explanation-panel}, firstnumber=1]{chapter/chapter_4/java/exp/ExplanationPanel.java}

Erklärung des ExplanationPanels

Nach der Erklärung auf das weitere Vorgehen eingehen...
Das weitere Vorgehen umfasst dabei besonders die Subkomponenten (TextPanel u. ImagePanel), zwischen denen hin und her gewechselt werden kann...
Hier ist es sinnvoll darauf einzugehen, wie die Reihenfolge aussieht: Vorab 1. TextPanel und dann 2. ImagePanel.
Grund für diese Ordnung ist die Art wie die Erklärungen (textuell und visuell) generiert werden
Auf die Anmerkung des in Reihe-Schalten in der Modellierung eingehen...
Generierung einer visuellen Erklärung setzt das Generieren einer texuellen Erklärung voraus.
Das Generieren einer textuellen Erklärung ein wichtiger Zwischenschritt in der Generierung einer visuellen Erklärung
Daher ist es sinnvoll erst TextPanel und die Generierung einer textuellen Erklärung vorwegzunehmen und dann auf die Ähnlichkeit bzw. Analogheit des ImagePanels und die Generierung einer visuellen Erklärung einzugehen...

\lstinputlisting[style=java-code, caption={TextPanel-Klasse}, label={sec4:impl:par:ui-elements:lst:text-panel}, firstnumber=1]{chapter/chapter_4/java/exp/text/TextPanel.java}

\lstinputlisting[style=java-code, caption={ImagePanel-Klasse}, label={sec4:impl:par:ui-elements:lst:image-panel}, firstnumber=1]{chapter/chapter_4/java/exp/img/ImagePanel.java}

% Für TextPanel auf Analogheit eingehen... TextPanel ist im Aufbau seiner Benutzeroberfläche, mit Außnahme der erweiterten Optionen und dem Textbereich im Vergleich zum JLabel + Icon, identisch...
% Ein weiterer Unterschied besteht im Generieren der Erklärung (textuell) in der nur der erste Schritt

% TextPanel und ImagePanel. Also vertauschen... Ist deshalb sinnvoll, weil das Generieren einer textuellen Erklärung ein wichtiger Zwischenschritt in der Generierung einer visuellen Erklärung ist... Hierzu auf das in Reihe Schalten aus der Modellierung eingehen...

% Grundbereich ExplanationPanel abgeschlossen
% Übergang zum letzten Grundbereich Nr.3 ExplainerConsole
% Simpler Grundbereich dient zum Loggen von Informationen
\lstinputlisting[style=java-code, caption={ExplainerConsole-Klasse}, label={sec4:impl:par:ui-elements:lst:explainer-console}, firstnumber=1]{chapter/chapter_4/java/exc/ExplainerConsole.java}

% Kurz vor Ende dieses Abschnitts die Zusammenfassung der Grundbereiche ansprechen und wie dies geschieht (initComponents) in ExplainerFrame.
% Methode initComponents als Listing zeigen und einfach nur anmerken, dass hier die Grundbereiche zusammengeschlossen werden
% Auch auf die besondere Abhängigkeit zu SwingX und JXMultiSplitPane eingehen
% Am Ende Grundbereiche zusammenfassen, zusammenschließen und auf Gesamtabbildung der Benutzeroberfläche in der Zusammenfassung verweisen...

% Dann ganz am Ende Bezug zum Forschungsziel anführen bzw. rückführen...
% Dann Überleitung zur Interaktion...

\FloatBarrier

\paragraph{Interaktion mit der Benutzeroberfläche}
\label{sec4:impl:par:ui-interaction}
% Hier dann die Komponenten zur Kontrolle der Interaktion vorstellen und dann anhand der Eigenschaften der Oberflächen Logik einfügen und beschreiben. (Nur Logik bezüglich der Interaktion)
% logik bezüglich der Verarbeitung bleibt noch offen und wird dann in FZ 2.3/I behandelt. (somit darauf verweisen)

% Auf Anwendungsfälle, Mechanismen und Sequenzdiagramme verweisen

% Erst für die Anwendungsfälle 1 bis 4, 5?

\lstinputlisting[style=java-code, caption={ImportGraphCodesController-Klasse}, label={sec4:impl:par:ui-elements:lst:import-gcs}, firstnumber=1]{chapter/chapter_4/java/interaction/ImportGraphCodesController.java}

\lstinputlisting[style=java-code, caption={RemoveSelectedGraphCodesController-Klasse}, label={sec4:impl:par:ui-elements:lst:remove-gcs}, firstnumber=1]{chapter/chapter_4/java/interaction/RemoveSelectedGraphCodesController.java}

\lstinputlisting[style=java-code, caption={GraphCodeSelectionListener-Klasse}, label={sec4:impl:par:ui-elements:lst:select-gcs}, firstnumber=1]{chapter/chapter_4/java/interaction/GraphCodeSelectionListener.java}

\lstinputlisting[style=java-code, caption={GraphCodeListMouseAdapter-Klasse}, label={sec4:impl:par:ui-elements:lst:mouse-gcs}, firstnumber=1]{chapter/chapter_4/java/interaction/GraphCodeListMouseAdapter.java}

\lstinputlisting[style=java-code, caption={GraphCodeCalculationController-Klasse}, label={sec4:impl:par:ui-elements:lst:calculate-gcs}, firstnumber=1]{chapter/chapter_4/java/interaction/GraphCodeCalculationController.java}

\clearpage

\subsubsection{Diskussion}

\clearpage


\clearpage

\subsection{FZ 2.3/I Integration generativer KI in das GMAF}
\label{sec4:impl:subsec:fz-integration}

In diesem Abschnitt wird die prototypische Proof-of-Concept Implementierung der \enquote{Integration der Endpunkte in das GMAF} beschrieben und behandelt.
In \cref{sec3:model} wurden Konzepte entwickelt und vorgestellt, die mögliche Vorgehensweisen zur Transformation von Graph Codes und Integration der Endpunkte der Schnittstelle von OpenAI beschreiben.
Die Beschreibung der Implementierung der Transformation von Graph Codes in eine Prompt wird im Folgenden in \cref{sec4:impl:subsubsec:gc-transformation} vorgenommen.
Des Weiteren wird in \cref{sec4:impl:subsubsec:endpoint-integration} die Integration und Interaktion mit den Endpunkten der Schnittstelle von OpenAI zum Erstellen von Erklärungen behandelt und beschrieben.

\subsubsection{Transformation von Graph Codes}
\label{sec4:impl:subsubsec:gc-transformation}
In diesem Abschnitt wird die Implementierung der Transformation der in Graph Codes gespeicherten Informationen beschrieben.
Im Folgenden zeigt \cref{sec3:model:subsubsec:gc-transformation:fig:set-up-prompt} die Methode \textit{setUpPrompt}, welche bereits in \cref{sec4:impl:par:ui-elements:lst:text-panel-p3,sec4:impl:par:ui-elements:lst:image-panel} angedeutet wurde und welche die Transformation von Graph Codes umsetzt und die zur Erstellung von Anfragen notwendige Prompt aufbereitet bzw. erstellt.

\lstinputlisting[style=java-code, caption={Quellcode für Methode \textit{setUpPrompt}.}, label={sec4:impl:subsubsec:gc-transformation:lst:set-up-prompt}, firstnumber=1]{chapter/chapter_4/java/methods/setUpPrompt.java}

Die für eine Anfrage notwendige Prompt wird durch einen Dialog dargestellt, der aus einer Reihe an Textnachricht besteht.
Innerhalb der Methode \textit{setUpPrompt} werden nun eine Reihe an Textnachrichten, die diesen Dialog erzeugen sollen, zusammengefügt.
Hierbei inkorporieren einige dieser Textnachrichten in einem geeigneten Textformat Informationen des entsprechenden Graph Codes.
Diese Informationen werden durch die Methoden \textit{listTerms} und \textit{getFormattedTerms} in ein geeignetes Textformat überführt und in die Textnachrichten eingebunden.
\cref{sec4:impl:subsubsec:gc-transformation:tcb:messages} zeigt diese Reihe an Textnachrichten.

\begin{messages}[breakable,colbacktitle=white, coltitle=black,label={sec4:impl:subsubsec:gc-transformation:tcb:messages}]{Textnachrichten für die Prompt}
  \textbf{System:} You are an assistant, who is able to generate cohesive textual explanations based on a collection of words.
  \tcbline
  \textbf{System:} The collection of words represents a dictionary.
  The dictionary contains so-called feature vocabulary terms.
  Additionally some of these terms are connected through a relationship.
  These relationships will be noted as $<i_t> - <i_{t_1},...,i_{t_n}>$, where $i_t$ denotes the index of a feature vocabulary term in the given collection.
  \tcbline
  \textbf{System:} Using these terms, we can create a coherent explanation that accurately describes the terms and its relations.
  \med
  An example could be: The image shows water, the sky, and clouds.
  We can imagine a scene with clouds floating in the sky above.
  \tcbline
  \textbf{User:} The collections of words is as follows: $<\textbf{terms}>$.
  Only respect these terms and its relations: $<\textbf{formats}>$, and ignore all others.
  Do not create an explanation regarding the dictionary.
  Only generate a text containing the terms of the dictionary like in the example above.
  \tcbline
  \textbf{Assistant:} Based on the dictionary, here is a cohesive text containing the terms from the dictionary:
\end{messages}

Die drei Parteien \textit{System}, \textit{Assistant} und \textit{User} sind Teil des Dialogs.
Die ersten drei Nachrichten stammen von der Partei \textit{System} und vermitteln dem GPT-Modell, wie es sich zu verhalten und zu antworten hat.
Darüber hinaus erhalten diese Nachrichten wichtige Informationen über die Struktur und Logik von Graph Codes mit, sowie die Form bzw. das Format in welchem diese Informationen dem Modell mitgeteilt werden.
Die vierte Nachricht stammt von der Partei \textit{User} und teilt dem Modell die spezifischen Informationen eines Graph Codes mit.
Diese Informationen sind eine Aufzählung der Merkmale im Wörterbuch eines Graph Codes, generiert durch die Methode \textit{listTerms}, sowie die Beziehungen zwischen den Merkmalen in Form eines geeigneten Textformats, generiert durch die Methode \textit{getFormattedTerms}.
Diese beiden Methoden werden durch die Klasse \textit{GraphCode} bereitgestellt und sind somit dem Bereich \textit{Model} zuzuweisen.
Die Methode \textit{listTerms} fusioniert die Merkmale des Wörterbuchs Mithilfe der Methode \textit{join} der String-Klasse (\textit{String.join(",", dictionary)}).
Die Methode \textit{getFormattedTerms} ist hingegen wesentlich komplexer und kann in \cref{sec4:impl:subsubsec:gc-transformation:lst:formatted-terms} eingesehen werden.
Die letzte Nachricht stammt von der Partei \textit{Assistant} und stellt einen unvollständigen Satz dar.
Die Aufgabe des Assistenten ist es auf den Nutzer zu antworten.
Durch die ersten Mitteilungen zum System weiß das Modell wie und auf welche Art es dem Nutzer zu antworten hat.
Der Assistent antwortet dabei auf die vom Nutzer eingegebenen Informationen in vom System vorgegebener Art und vervollständigt den unvollständigen Satz mit einer Erklärung in Abhängigkeit der vom Nutzer angegebenen Informationen.

\lstinputlisting[style=java-code, caption={Quellcode für Methode \textit{getFormattedTerms}.}, label={sec4:impl:subsubsec:gc-transformation:lst:formatted-terms}, firstnumber=1]{chapter/chapter_4/java/methods/formatted-terms.java}

Um die Beziehungen zwischen Merkmalen in einem Graph Code geeignet darzustellen, werden mit dem Algorithmus \textit{getFormattedTerms} Textformate generiert.
Hierfür durchläuft der Algorithmus einzeln die Reihen der zweidimensionalen, quadratischen Matrix und sammelt für jeden einzigartigen Wert eine Reihe an Merkmalen.
Hierbei werden allerdings diagonale Einträge und Einträge mit dem Wert null ausgelassen.
Dies bedeutet, dass jeder einzigartige Wert in einer Reihe mit mehreren Merkmalen assoziiert sein kann.
Zu diesem Zweck wird die Klasse \textit{MultiMap} der Abhängigkeit \textit{Guava} verwendet.
Diese hält für einen jeweiligen Schlüssel, welcher ein einzigartiger Wert sein wird, eine Kollektion an Elementen, in diesem Fall Merkmale.
Die Schlüssel werden numerisch sortiert (\textit{treeKeys}) und die in die Kollektion hinzugefügten Merkmale werden nach Reihenfolge ihres Hinzufügens sortiert (\textit{arrayListValues}).
Abschließend werden die Einträge in der \textit{MultiMap} durchlaufen und es werden Textformate generiert, indem die Werte in einem Format (\textit{String.format()}) zusammengefügt werden.
Anstatt allerdings die Merkmale zu fusionieren, werden nur die Indizes der Merkmale im Wörterbuch ermittelt und zusammengefügt und dann dem Format beigefügt.
Ergebnis ist eine Menge an Formaten für eine Reihe in der Matrix, die dann am Ende fusioniert werden.
Schlussendlich werden alle fusionierten Formate in einem einzigen Text zusammengefügt und als Ergebnis ausgegeben.
Die Implementierung der Transformation der in der Matrix enthaltenen Informationen entspricht somit dem in der Modellierung in \cref{sec3:model:subsubsec:gc-transformation} entwickelten Konzept zur Transformation der Matrix.

Folgendes Beispiel zeigt die transformierten Informationen für einen echten Graph Code:
Die Aufzählung des Vokabular lautet: \enquote{root-image, water, sky, cloud, building, skyscraper, dusk, body of water, lake, atmospheric phenomenon, horizon}.
Die in Textformat überführten Einträge der Matrix des Graph Codes lauten: $<1> - <2,3,4,5,6,7,8,9,10,11>$.
Eingefügt in \cref{sec4:impl:subsubsec:gc-transformation:tcb:messages} und als Anfrage an den Endpunkt Text gesendet, ergibt sich folgender Text als Antwort bzw. Erklärung:

\textit{The root-image captures a breathtaking view at dusk. The horizon is beautifully painted with hues of the setting sun, creating an atmospheric phenomenon that is a sight to behold. Dominating the scene are a skyscraper and other buildings, their silhouettes stark against the sky. Below, a body of water, perhaps a lake, mirrors the sky. The water, calm and serene, adds a sense of tranquility to the scene. Floating above are clouds, their edges tinged with the colors of dusk, adding depth to the sky.}

Weitere Beispiele für Graph Codes, transformierte Informationen und daraus generierten Erklärungen können in \cref{sec4:impl:subsec:summary} \enquote{Zusammenfassung} eingesehen werden.

\subsubsection{Integration der Endpunkte}
\label{sec4:impl:subsubsec:endpoint-integration}
In diesem Abschnitt wird die Implementierung der Integration der Endpunkte, sowie die Interaktion zur Erstellung einer Anfrage an diese Endpunkte behandelt.
Der Ablauf der Erklärung der Integration und Interaktion mit den Endpunkten der Schnittstelle entspricht dabei dem Ablauf der Modellierung und Implementierung, da für die Anfrage an den Endpunkt Bild zuerst eine Anfrage an den Endpunkt Text notwendig ist.

Im Folgenden zeigt \cref{sec4:impl:subsubsec:gc-transformation:lst:text-action} die Methode \textit{actionPerformed}, die Teil der Klasse \textit{TextPanel} ist und mit Anwendungsfall \hyperref[sec3:model:uc-1.8]{UC-1.8} assoziiert ist.
Diese Methode realisiert die Interaktion und damit auch die Integration der notwendigen Komponenten zum Erstellen einer Anfrage an den Endpunkt Text.

\lstinputlisting[style=java-code, caption={Anfrage an den Endpunkt Text.}, label={sec4:impl:subsubsec:gc-transformation:lst:text-action}, firstnumber=1]{chapter/chapter_4/java/methods/text-action.java}

In Zeile 3 wird eine Instanz der Schnittstelle \textit{OpenAiService} initialisiert und mit der Variable \textit{service} eingeführt.
Für die Initialisierung ist ein gültiger API-Schlüssel notwendig.
Des Weiteren wird ein Zeitüberschreitung von 60 Sekunden eingestellt.
In Zeile 4 wird dann ein ExecutorService zum Ausführen eines Tasks bzw. einer Aufgabe initialisiert und mit der Variable \textit{executor} eingeführt.
Die Aufgabe wird das Erstellen einer Anfrage an den Endpunkt Text sein und wird hierfür durch einen Thread dargestellt.
Die erste Aktion in diesem Thread ist das Erstellen der Komponente \textit{ChatCompletionRequest}, welche eine Anfrage zur Chatvervollständigung darstellt und mit der Variable \textit{chatComplReq} eingeführt wird.
Das Erstellen dieser Komponente erfolgt über die Komponente \textit{ChatCompletionRequestBuilder}, mit welcher eine Anfrage zur Chatvervollständigung Stück für Stück aufgebaut, parametrisiert und finalisiert werden kann.
Die Variable \textit{chatComplReq} enthält alle wichtigen Informationen zur Anfrage und kann im Weiteren über \textit{service} an den Endpunkt Text gesendet werden.
Das Ergebnis dieser Anfrage ist die Komponente \textit{ChatCompletionResult}, welche die Antwort des Endpunktes an die Anfrage darstellt und mit der Variable \textit{chatComplRes} eingeführt wird.
\textit{chatComplRes} wirkt somit als Container für die Antwortinformationen der Anfrage.
Diese Informationen, im Folgenden einfach nur die textuelle Erklärung, kann dann aus diesem Container entnommen und im Weiteren in einem Textfeld angezeigt werden.
Schlussendlich wird durch \textit{executor} die Aufgabe bzw. der Thread ausgeführt und nach Ausführung beendet.

\lstinputlisting[style=java-code, caption={Anfrage an den Endpunkt Bild.}, label={sec4:impl:subsubsec:gc-transformation:lst:image-action}, firstnumber=1]{chapter/chapter_4/java/methods/image-action.java}

\cref{sec4:impl:subsubsec:gc-transformation:lst:image-action} zeigt das Erstellen einer Anfrage an den Endpunkt Bild.
Für das Erstellen einer Anfrage an diesen Endpunkt ist zuerst eine Anfrage an den Endpunkt Text notwendig, um eine textuelle Erklärung, die als Bildbeschreibung für die visuelle Erklärung bzw. das zu erzeugende Bild fungieren soll, zu erzeugen.
Die Implementierung dieser Anfrage verhält sich analog zu der in \cref{sec4:impl:subsubsec:gc-transformation:lst:text-action}.
Allerdings bedarf es einer passenden Parametrisierung dieser Anfrage, da der Endpunkt Bild nur eine maximale Anzahl von 1000 Zeichen akzeptiert.
Durch diese Parametrisierung kann sichergestellt werden, dass die Antwort der ersten Anfrage an den Endpunkt Text eine maximale Länge von ~75 Token beträgt.
Im nächsten Schritt erfolgt das Erstellen einer Anfrage \textit{CreateImageRequest} durch die Komponente \textit{CreateImageRequestBuilder} und wird mit der Variable \textit{imgReq} eingeführt.
Die Antwort der vorherigen Anfrage ist dabei Teil der Parametrisierung dieser Anfrage.
Im nächsten Schritt wird die Anfrage über \textit{service} an den Endpunkt Bild gesendet.
Das Ergebnis dieser Anfrage ist die Komponente \textit{ImageResult}, welche als Container für die Antwortinformationen wirkt.
Eine dieser Information ist eine URL, die zum erzeugten Bild führt und in einem JLabel als Icon angezeigt werden kann.
Schlussendlich wird durch \textit{executor} die Aufgabe bzw. der Thread ausgeführt und nach Ausführung beendet.

In Bezug auf die \cref{sec4:impl:subsubsec:gc-transformation:lst:text-action,sec4:impl:subsubsec:gc-transformation:lst:image-action} kann festgehalten werden, dass die in der Implementierung der Interaktion und Integration der Endpunkte Text und Bild verwendeten Komponenten den im Mechanismus (siehe \cref{sec3:model:par:mechanism-use-cases:fig:mech-uc-1.8}) identifizierten Komponenten entsprechen.
Ebenso entspricht der Ablauf der Aktionen in der umgesetzten Interaktion dem im Sequenzdiagramm (siehe \cref{sec3:model:par:seq-use-cases:fig:seq-diag-uc-1.8}) beschriebenem Ablauf.

\subsubsection{Diskussion}
\label{sec4:impl:subsubsec:fz2:discussion}
Im ersten Forschungsziel wurden bereits beim Beschreiben der Implementierung die Funktionen \textit{setUpPrompt} und \textit{actionPerformed} der Klasse \textit{TextPanel} bzw. \textit{ImagePanel} angedeutet, aber aus Gründen der Zuordnung nicht näher beschrieben.
In diesem FZ wurde die Implementierung dieser Methoden, wie auch im Rahmen der Modellierung in den \cref{sec3:model:subsubsec:gc-transformation,sec3:model:subsubsec:genai-integration} vorgesehen, genauer beschrieben.

In \cref{sec4:impl:subsubsec:gc-transformation} wurde dabei die Implementierung der Transformation von Informationen in Graph Codes genauer beschrieben.
Dies umfasst eine Zusammenstellung von Textnachrichten in einem Dialog, die die Prompt bzw. die Eingabe in den Endpunkt Text darstellen, sowie die Methoden \textit{listTerms} und \textit{getFormattedTerms} zum Transformieren der in Graph Codes enthaltenen Informationen.
Die Implementierung dieser Methoden wurde dabei mit kompaktem Quellcode dokumentiert und beschrieben, sowie durch Beispiele verdeutlicht.
In \cref{sec4:impl:subsubsec:gc-transformation} wurde somit der Bereich \textit{Model} behandelt.

Des Weiteren wurde in \cref{sec4:impl:subsubsec:endpoint-integration} die Implementierung der Interaktion und damit auch die Integration der Endpunkte Text und Bild der Schnittstelle von OpenAI genauer beschrieben.
Dies umfasst das Initialisieren der Schnittstelle, das Erstellen und Parametrisieren der notwendigen Komponenten einer Anfrage, sowie das Senden und Verarbeiten der Anfrage an die Schnittstelle.
Hierfür wurden die Methoden mit kompaktem Quellcode dokumentiert und beschrieben.
In \cref{sec4:impl:subsubsec:endpoint-integration} wurde somit der verbleibende Teil des Bereichs \textit{Controller} behandelt.


\clearpage

\subsection{Zusammenfassung}
\label{sec4:impl:subsec:summary}
In diesem Kapitel wurde die Implementierung behandelt und es wurde in \cref{sec4:impl:subsec:fz-explainability} die prototypische Implementierung einer Benutzungsschnittstelle mit geeigneten Interaktionsmöglichkeiten zur Verarbeitung von Graph Codes und zur Erzeugung von textuellen und visuellen Erklärungen vorgestellt und beschrieben, sowie in \cref{sec4:impl:subsec:fz-integration} die prototypische Proof-of-Concept Implementierung von Funktionen zur Transformation von Graph Codes, sowie zur Interaktion und Integration der Endpunkte der Schnittstelle von OpenAI zur Erzeugung von textuellen und visuellen Erklärungen vorgestellt und beschrieben.
Im Folgenden werden die wesentlichen Ergebnisse und Erkenntnisse der Implementierung nach den Forschungszielen der Implementierungsphase, wie nach der Methodik nach Nunamaker vorgesehen, aufgeschlüsselt und in \cref{sec4:impl:subsubsec:summary-findings} kurz zusammengefasst.
Des Weiteren wird in \cref{sec4:impl:subsubsec:further-approach} das weitere Vorgehen festgehalten und schlussendlich in \cref{sec4:impl:subsec:summary:table:summary} eine Übersicht des aktuellen Arbeitsstands in einer Tabelle dargestellt.

\subsubsection{Gewonnene Erkenntnisse}
\label{sec4:impl:subsubsec:summary-findings}
In diesem Abschnitt werden die Erkenntnisse aus den Forschungszielen \enquote{FZ 1.3/I Erklärbarkeit von MMIR mittels generativer KI} und \enquote{FZ 2.3/I Integration generativer KI in das GMAF} zusammengefasst.

Im ersten Forschungsziel \enquote{FZ 1.3/I Erklärbarkeit von MMIR mittels generativer KI} wurde eine prototypische Proof-of-Concept Implementierung der in \cref{sec3:model} im Rahmen der Modellierung entwickelten Konzepte zur Erklärbarkeit von MMIR mittels generativer KI beschrieben.
Dies umfasst eine Beschreibung des Aufbaus der Benutzeroberfläche und der in ihr verbauten Komponenten.
Des Weiteren wurde die Interaktion zwischen Benutzern des GMAF und diesen Komponenten, sowie die Interaktion zwischen diesen Komponenten thematisiert.

Im zweiten Forschungsziel \enquote{FZ 2.3/I Integration generativer KI in das GMAF} wurde eine prototypische Proof-of-Concept Implementierung der in \cref{sec3:model} im Rahmen der Modellierung entwickelten Konzepte zur Integration generativer KI in das GMAF beschrieben.
Dies umfasst eine Beschreibung der Funktionen zur Transformation der in Graph Codes gespeicherten Informationen, sowie eine Beschreibung der Interaktion mit und Integration von Endpunkten der Schnittstelle.

Die folgenden \cref{sec4:impl:subsubsec:summary-findings:fig:ui-ex-1,sec4:impl:subsubsec:summary-findings:fig:ui-ex-2,sec4:impl:subsubsec:summary-findings:fig:ui-ex-3,sec4:impl:subsubsec:summary-findings:fig:ui-ex-4,sec4:impl:subsubsec:summary-findings:fig:ui-ex-5} verdeutlichen die Erkenntnisse bzw. Ergebnisse der prototypischen Proof-of-Concept Implementierung anhand Screenshots.
\cref{sec4:impl:subsubsec:summary-findings:fig:ui-ex-1,sec4:impl:subsubsec:summary-findings:fig:ui-ex-2} zeigen dabei den Ausgangspunkt des Programms zur Erklärung von Graph Codes.
\cref{sec4:impl:subsubsec:summary-findings:fig:ui-ex-1} zeigt hierbei in der rechten Arbeitsfläche \textit{ExplanationPanel} die Schnittstelle zur Erzeugung von visuellen Erklärungen bzw. Bildern für Graph Codes und \cref{sec4:impl:subsubsec:summary-findings:fig:ui-ex-2} entsprechend die Schnittstelle zur Erzeugung von textuellen Erklärungen.
In keiner der beiden Abbildungen wurde zuvor ein Graph Code ausgewählt, wie es auch anhand des Platzhalters \enquote{No GraphCode selected!} im rechten Teil der linken Arbeitsfläche bzw. dem \textit{GraphCodeTable} erkennbar ist.

% In \cref{sec4:impl:subsubsec:summary-findings:fig:ui-ex-3} wurden in die Liste in der linken Arbeitsfläche eine Reihe an Graph Codes importiert. \cref{sec4:impl:subsubsec:summary-findings:fig:ui-ex-4} zeigt daraufhin die Auswahl eines Graph Codes aus der Liste, den Einfluss, die diese Auswahl auf die anderen Komponenten, sowie eine durch die Schnittstelle generierte, visuelle Erklärung bzw. Bild zum ausgewählten Graph Code. Besonders beeinflusst wird die Komponente \textit{GraphCodeTable}, in welcher der ausgewählte Graph Code in Tabellenform dargestellt wird und die Komponente \textit{ExplanationPanel}, in welcher die aus den Informationen des Graph Codes erzeugte Prompt in einem Textfeld \enquote{Prompt} dargestellt wird. Das mit dieser Prompt generierte Bild ist im unteren Teil von \textit{ExplanationPanel} dargestellt. Analog zeigt \cref{sec4:impl:subsubsec:summary-findings:fig:ui-ex-5} die Auswahl eines anderen Graph Codes aus der Liste und eine in \textit{ExplanationPanel} generierte, textuelle Erklärung.

\cref{sec4:impl:subsubsec:summary-findings:fig:ui-ex-4,sec4:impl:subsubsec:summary-findings:fig:ui-ex-5} zeigen daraufhin jeweils die Auswahl eines Graph Codes aus der Liste und wie diese Auswahl die anderen Komponenten der Schnittstelle beeinflusst, sowie eine für den entsprechend gewählten Erklärungstypen generierte Erklärung.
Die beeinflussten Komponenten sind dabei \textit{GraphCodeTable}, in welcher der ausgewählte Graph Code in Tabellenform dargestellt wird und \textit{ExplanationPanel}, in welcher die aus den Informationen des Graph Codes erzeugte Prompt in einem Textfeld \enquote{Prompt} dargestellt wird.
Die aus dieser Prompt generierte Erklärung ist ensprechend im unteren Teil von \textit{ExplanationPanel} in entweder einem Textfeld für die textuelle Erklärung, oder in einem Bild für die visuelle Erklärung dargestellt.

% Weiter auf die Screenshots eingehen...
% Zu Anfang jeder Beschreibung einfach beschreiben was es im Screenshot zu sehen gibt.
% Dann eventuell auch recht detailliert beschreiben, in welchem Zustand die Komponenten (Liste, Table, ExplanationPanel) jeweils sind.

\begin{landscape}
  \begin{figure}
    \includegraphics[width=\textwidth]{chapter/chapter_4/ui-ex-1}
    \caption{Benutzungsschnittstelle zum Generieren einer visuellen Erklärung.}
    \label{sec4:impl:subsubsec:summary-findings:fig:ui-ex-1}
  \end{figure}
\end{landscape}

\begin{landscape}
  \begin{figure}
    \includegraphics[width=\textwidth]{chapter/chapter_4/ui-ex-2}
    \caption{Benutzungsschnittstelle zum Generieren einer textuellen Erklärung.}
    \label{sec4:impl:subsubsec:summary-findings:fig:ui-ex-2}
  \end{figure}
\end{landscape}

\begin{landscape}
  \begin{figure}
    \includegraphics[width=\textwidth]{chapter/chapter_4/ui-ex-3}
    \caption{Liste mit einer Reihe an importieren Graph Codes.}
    \label{sec4:impl:subsubsec:summary-findings:fig:ui-ex-3}
  \end{figure}
\end{landscape}

\begin{landscape}
  \begin{figure}
    \includegraphics[width=\textwidth]{chapter/chapter_4/ui-ex-4}
    \caption{Auswahl eines Graph Codes und eine generierte, visuellen Erklärung bzw. ein Bild.}
    \label{sec4:impl:subsubsec:summary-findings:fig:ui-ex-4}
  \end{figure}
\end{landscape}

\begin{landscape}
  \begin{figure}
    \includegraphics[width=\textwidth]{chapter/chapter_4/ui-ex-5}
    \caption{Auswahl eines Graph Codes und eine generierte, textuelle Erklärung.}
    \label{sec4:impl:subsubsec:summary-findings:fig:ui-ex-5}
  \end{figure}
\end{landscape}

\FloatBarrier

\subsubsection{Weiteres Vorgehen}
\label{sec4:impl:subsubsec:further-approach}
Im nachfolgenden Kapitel wird die Evaluierung der in diesem Kapitel beschriebenen, prototypischen Proof-of-Concept Implementierung vorgenommen.
Genauer wird in \cref{sec5:eval} in \hyperref[sec5:eval:subsec:fz-explainability]{FZ 1.4/E} eine Evaluierung, der im Rahmen der prototypischen Proof-of-Concept Implementierung umgesetzten Benutzungsschnittstelle und der Interaktion mit dieser, vorgenommen.
In \hyperref[sec5:eval:subsec:fz-integration]{FZ 2.4/E} wird dann eine Evaluierung, der im Rahmen der prototypischen Proof-of-Concept Implementierung umgesetzten Funktionen zur Transformation der in Graph Codes enthaltenen Informationen, vorgenommen.

Die nachfolgende Tabelle stellt eine Erweiterung der \cref{sec3:model:subsec:summary:table:summary} dar und gibt eine Übersicht über den aktuellen Arbeitsstand nach Abschluss dieses Kapitels.

\begingroup
\def\arraystretch{1.1}%
\begin{xltabular}{\linewidth}{
            @{}
            >{
                \hsize=0.2\linewidth
                \raggedright\arraybackslash
            }X
            >{
                \hsize=0.6\linewidth
                \raggedright\arraybackslash
            }X
            >{
                \hsize=0.2\linewidth
            }X
            @{}
        }

        % First Header

        \caption{Tabelle zur Übersicht des aktuellen Arbeitsstands.}
        \label{sec4:impl:subsec:summary:table:summary}
        \\

        \toprule
        \multicolumn{3}{
            >{
                    \hsize=\linewidth\centering\arraybackslash
            }X
        }
        {
            \textbf{Forschungsziele}
        } \\ \midrule
        \textbf{FZ / OH} &  \textbf{Beschreibung} & \textbf{Referenz} \\ \midrule

        \endfirsthead

        \toprule
        \multicolumn{3}{
            >{
                    \hsize=\linewidth\centering\arraybackslash
            }X
        }
        {
            \textbf{Forschungsziele}
        } \\ \midrule
        \textbf{FZ / OH} & \textbf{Beschreibung} & \textbf{Referenz} \\ \midrule

        \endhead

        % Lower Rows

        \multicolumn{3}{
            >{
                    \hsize=\linewidth\centering\arraybackslash
            }X
        }
        {
            \textbf{Erklärbarkeit von MMIR mittels generativer KI}
        }
        \\
        \midrule

        FZ 1.1/O
        &
        Recherche zur Erklärbarkeit von MMIR mittels generativer KI
        \\

        &
        Grundlegende Technologien:
        &

        \\

        &
        \tabitem GMAF
        &
        \cref{sec2:sota:subsubsec:gmaf}
        \\

        &
        \tabitem MMFG
        &
        \cref{sec2:sota:subsubsec:mmfg}
        \\

        &
        \tabitem Graph Code
        &
        \cref{sec2:sota:subsubsec:graph-codes}
        \\

        % Offene Herausforderungen aus FZ1/O

        OH 1.1
        &
        Erste offene Herausforderung
        &
        \hyperref[sec2:sota:oi:1.1]{\textbf{OH 1.1}}
        \\


        &
        Systeme generativer KI und ein Überlick über aktuelle Systeme
        &
        \cref{sec2:sota:subsubsec:genai}
        \\

        &
        Diskussion und Auswahl von Systemen
        &
        \cref{sec2:sota:subsubsec:fz1:discussion}
        \\

        OH 1.2
        &
        Zweite offene Herausforderung
        &
        \hyperref[sec2:sota:oi:2.1]{\textbf{OH 2.1}}
        \\

        \midrule

        FZ 1.2/TB
        &
        Modellierung der Erklärbarkeit von MMIR mittels generativer KI
        &

        \\

        &
        Erklärbarkeit durch generative KI
        &
        \cref{sec3:model:subsubsec:explainability-through-genai}
        \\

        &
        $\rightarrow$ Behandlung der ersten offenen Herausforderung \hyperref[sec2:sota:oi:1.1]{\textbf{OH 1.1}}
        &
        \\

        &
        Anwendungsfälle:
        &
        \cref{sec3:model:subsubsec:use-cases}
        \\

        &
        \tabitem Textuelle Beschreibungen
        &
        %\cref{sec3:model:par:textual-desc-use-cases}
        \\

        &
        Wireframes
        &
        \cref{sec3:model:par:wireframe}
        \\

        &
        Mechanismen
        &
        \cref{sec3:model:par:mechanism-use-cases}
        \\

        &
        Sequenzdiagramme
        &
        \cref{sec3:model:par:seq-use-cases}
        \\

        &
        $\rightarrow$ Behandlung der zweiten offenen Herausforderung \hyperref[sec2:sota:oi:1.2]{\textbf{OH 1.2}}
        &
        \\

        \midrule

        FZ 1.3/I
        &
        Implementierung der Erklärbarkeit von MMIR mittels generativer KI
        &

        \\

        &
        Benutzungsschnittstelle:
        &
        \cref{sec4:impl:subsubsec:ui}
        \\

        &
        \tabitem Elemente der Benutzeroberfläche
        &
        \\

        &
        \tabitem Interaktion mit der Benutzeroberfläche
        &
        \\

        \midrule

        FZ 1.4/E
        &
        Evaluierung der Erklärbarkeit von MMIR mittels generativer KI
        &

        \\

        \midrule

        \multicolumn{3}{
            >{
                    \hsize=\linewidth\centering\arraybackslash
            }X
        }
        {
            \textbf{Integration generativer KI in das GMAF}
        }
        \\
        \midrule

        FZ 2.1/O
        &
        Recherche zur Integration generativer KI in das GMAF
        &

        \\


        &
        Aufzeigen der Integrationsmöglichkeiten von:
        &

        \\

        &
        \tabitem Graph Codes
        &
        \cref{sec2:sota:subsubsec:gc-capabilities-integration}
        \\

        &
        Erste offene Herausforderung
        &
        \hyperref[sec2:sota:oi:2.1]{\textbf{OH 2.1}}
        \\

        &
        \tabitem Systemen generativer KI
        &
        \cref{sec2:sota:subsubsec:genai-capabilities-integration}
        \\

        &
        Zweite offene Herausforderung
        &
        \hyperref[sec2:sota:oi:2.2]{\textbf{OH 2.2}}
        \\

        \midrule

        FZ 2.2/TB
        &
        Modellierung der Integration generativer KI in das GMAF
        &

        \\

        &
        Transformation von Graph Codes
        &
        \cref{sec3:model:subsubsec:gc-transformation}
        \\

        &
        \tabitem Transformation des Vokabulars
        &
        \\

        &
        \tabitem Transformation der Matrix
        &
        \\

        &
        \tabitem Anwendung von Graph Code Metriken
        &
        \\

        &
        $\rightarrow$ Behandlung der ersten offenen Herausforderung \hyperref[sec2:sota:oi:2.1]{\textbf{OH 2.1}}
        &
        \\

        &
        Einbindung generativer KI in das GMAF
        &
        \cref{sec3:model:subsubsec:genai-integration}
        \\

        &
        $\rightarrow$ Behandlung der zweiten offenen Herausforderung \hyperref[sec2:sota:oi:2.2]{\textbf{OH 2.2}}
        &
        \\

        \midrule

        FZ 2.3/I
        &
        Implementierung der Integration generativer KI in das GMAF
        &

        \\

        &
        Transformation von Graph Codes
        &
        \cref{sec4:impl:subsubsec:gc-transformation}
        \\

        &
        Integration der Endpunkte
        &
        \cref{sec4:impl:subsubsec:endpoint-integration}
        \\

        \midrule

        FZ 2.4/E
        &
        Evaluierung der Integration generativer KI in das GMAF
        &

        \\

        \bottomrule
\end{xltabular}
\endgroup

