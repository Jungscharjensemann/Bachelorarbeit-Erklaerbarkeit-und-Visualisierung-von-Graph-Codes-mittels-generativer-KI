\section{Implementierung}
\label{sec4:impl}
Dieses Kapitel widmet sich der Implementierung und es werden, entsprechend der Methodik von Nunamaker \cite{nunamaker}, alle Forschungsziele der Implementierungsphase behandelt.
Ziel der Implementierung ist die Validierung der in \cref{sec3:model} beschriebenen Modellierung, und die Vorbereitung einer detaillierten Evaluierung in der Experimentphase, sowie Untersuchungen weiterer Beobachtungen, die während der Implementierung gemacht wurden.
Die Struktur dieses Kapitels folgt der logischen Abfolge der Forschungsziele der Implementierungsphase.
Doch zuvor werden in \cref{sec4:impl:basics} die Grundlagen bzw. Vorbereitungen für die Implementierung getroffen.
Dies umfasst die Beschreibung der geplanten Struktur und Architektur der Implementierung, sowie weitere technische Voraussetzungen und die in dieser prototypischen Implementierung genutzten Bibliotheken bzw. Abhängigkeiten.
In \cref{sec4:impl:subsec:fz-explainability} werden ausgewählte Implementierungsdetails der Erklärbarkeit von MMIR mittels generativer KI vorgestellt.
% Dies umfasst...
Analog dazu werden in \cref{sec4:impl:subsec:fz-integration} ausgewählte Implementierungsdetails der Integration generativer KI in das GMAF vorgestellt.
% Dies umfasst...

\subsection{Grundlagen der Implementierung}
\label{sec4:impl:basics}
In diesem Abschnitt werden die Grundlagen für die in diesem Kapitel beschriebene prototypische Proof-of-Concept Implementierung festgehalten.
Dies umfasst die allgemeine Architektur, ihr zugrunde liegendes Prinzip, ihr hierarchischer Aufbau, sowie die Abhängigkeiten zu Drittanbieter-Bibliotheken.

\paragraph{Architekturmuster}
Die Implementierung der Software folgt dem Muster des \textit{Model-View-Controller}-Prinzips und teilt die Architektur der Software in drei seperate Bereiche: \textit{Model} für die Datenstrukturen, \textit{View} für das Frontend bzw. die Benutzerschnittstellen und \textit{Controller} zum Handhaben von Ereignissen und Interaktionen zwischen den Bereichen \textit{Model} und \textit{View}.
Ziel dieser Architektur ist durch das gezielte Trennen dieser Bereiche die Eigenschaften wie Wiederverwendbarkeit, Wartbarkeit und Robustheit der Software zu erhöhen und zu stärken.

Das erste Forschungsziel \hyperref[sec4:impl:subsec:fz-explainability]{FZ 1.3/I} widmet sich ... und deckt somit die Bereiche \textit{View, Controller} ab.
Das zweite Forschungsziel \hyperref[sec4:impl:subsec:fz-integration]{FZ 2.3/I} widmet sich ... und deckt somit den Bereich \textit{Model} ab.

\cref{sec4:impl:subsec:basiscs:fig:hierarchy} zeigt die Hierarchie des Quellcodes der in diesem Kapitel beschriebenen prototypischen Proof-of-Concept Implementierung.

\def\Size{4pt}
\tikzset{
      folder/.pic={
        \filldraw[draw=folderborder,top color=folderbg!50,bottom color=folderbg]
          (-1.05*\Size,0.2\Size+5pt) rectangle ++(.75*\Size,-0.2\Size-5pt);  
        \filldraw[draw=folderborder,top color=folderbg!50,bottom color=folderbg]
          (-1.15*\Size,-\Size) rectangle (1.15*\Size,\Size);
      }
    }

\begin{figure}[htb]
\begin{forest}
      for tree={
        font=\ttfamily,
        grow'=0,
        child anchor=west,
        parent anchor=south,
        anchor=west,
        calign=first,
        inner xsep=7pt,
        edge path={
          \noexpand\path [draw, \forestoption{edge}]
          (!u.south west) +(7.5pt,0) |- (.child anchor) pic {folder} \forestoption{edge label};
        },
        % style for your file node 
        file/.style={edge path={\noexpand\path [draw, \forestoption{edge}]
          (!u.south west) +(7.5pt,0) |- (.child anchor) \forestoption{edge label};},
          inner xsep=2pt,font=\small\ttfamily
                     },
        before typesetting nodes={
          if n=1
            {insert before={[,phantom]}}
            {}
        },
        fit=band,
        before computing xy={l=15pt},
      }  
    [de
        [ja
            [controller]
            [model]
            [view]
        ]
        [swa]
    ]
\end{forest}
\caption{Hierarchie des Quellcodes.}
\label{sec4:impl:subsec:basiscs:fig:hierarchy}
\end{figure}

\paragraph{Abhängigkeiten (Maven)}
Für die prototypische Proof-of-Concept Implementierung der in \cref{sec3:model} vorgestellten konzeptuellen Modellierung wird die Programmiersprache Java \cite{java} ausgewählt.
Der in diesem Kapitel vorgestellte Quellcode ist auf GitHub verfügbar.
Die Implementierung in Java baut hierbei auf Maven \cite{maven}, einem Build-Tool der Apache Software Foundation für die Projektverwaltung und Management von Abhängigkeiten, auf.

\clearpage

\subsection[FZ 1.3/I Erklärbarkeit von MMIR mittels generativer KI]{\texorpdfstring{FZ 1.2/TB Erklärbarkeit von MMIR mittels \\ generativer KI}{FZ 1.3/I Erklärbarkeit von MMIR mittels generativer KI}}
\label{sec4:impl:subsec:fz-explainability}

\clearpage

\subsection{FZ 2.3/I Integration generativer KI in das GMAF}
\label{sec4:impl:subsec:fz-integration}

\clearpage

\subsection{Zusammenfassung}
\label{sec4:impl:subsec:summary}