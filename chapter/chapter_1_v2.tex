\section{Einleitung}
\label{sec1:intro}
\begin{itemize}
    \item Einleitung soll recht allgemein ins Thema einführen.
    \item Konkreter Anwendungsfall, o. Problem einsteigen.
    \item Leser muss sich mit dem Thema identifizieren können.
    \item Abschnitt 2 Forschungsthema findet eher in Kap. 2 Platz, da diese Informationen erst nach einer ausführlichen und tiefergehenden Recherche bekannt sind.
    \item In Rolle eines Außenstehenden versetzen und überlegen, wie ein Leser \enquote{abgeholt} werden kann.
    \item Einleitung muss für jedermann verständlich sein!
    \item Einleitung möglichst ohne Fachbegriffe / technologie-agnostisch
\end{itemize}

\subsection{Motivation}
\label{sec1:intro:subsec:motivation}
\begin{itemize}
    \item Motivation nirgends zu finden.
    \item Warum schreibe ich diese Arbeit überhaupt?
    \item Wo ist diese Arbeit verortet?
    \item Hier wäre der allererste Abschnitt ganz passend.
    \item Was interessiert mich persönlich an diesem Thema?
    \item Es handelt sich um eine BA am Lehrgebiet ... von M. Hemmje.
    \item Thema als solches besser herausarbeiten
\end{itemize}

\subsection{Problembeschreibung}
\label{sec1:intro:subsec:problems}
\begin{itemize}
    \item Beispiel anführen u. mittels Screenshots zeigen, wo aktuell Forschungslücken bestehen.
    \item Wo ist der Mehrwert für Benutzer?
    \item Auch Problembeschreibungen sind schon viel zu konkret.
    \item PBs schwierig, da große Überlappungen -> schärfere Trennung der Probleme
\end{itemize}

\subsection{Forschungsfragen}
\label{sec1:intro:subsec:research-questions}

\subsection{Methodik und Ziele}
\label{sec1:intro:subsec:methodology-goals}
\begin{figure}[htb]
    \centering
    \resizebox*{0.75\textwidth}{!}{
        \begin{minipage}{\textwidth}
            \begin{tcolorbox}[
                enhanced, width=\textwidth, height=\textwidth, colback=white, colframe=black,
                overlay={
                    \tikzset{
                        exstyle/.style={-{Triangle[angle=90:6pt,length=3mm,fill=black]}}
                    }
                    \def\rad{\textwidth-2.5cm}
                    \node [circle, minimum size=\rad] (c) at ([yshift=-0.8cm]frame.center) {};
                    \def\labellist{
                          \shortstack{
                            \Large \textbf{Beobachtung} \\
                            \rule{3cm}{1pt} \\
                            \footnotesize
                            Umfragestudien, \\ 
                            \footnotesize
                            Fallstudien, \\
                            \footnotesize
                            Feldforschung
                          }, 
                          \shortstack{
                            \Large \textbf{Experiment} \\
                            \rule{3cm}{1pt} \\
                            \footnotesize
                            Computersimulationen,\\ 
                            \footnotesize
                            Feldexperimente, \\
                            \footnotesize
                            Laborexperimente
                          },
                          \shortstack{
                            \Large \textbf{Theorie} \\ \Large \textbf{Bildung} \\
                            \rule{3cm}{1pt} \\
                            \footnotesize
                            konzept. Rahmen-\\ 
                            \footnotesize
                            bedingungen, mathe\\
                            \footnotesize
                            matische Modelle\\
                            \footnotesize 
                            Methoden
                          }
                    }
                    
                    \foreach [count=\i] \x in \labellist {
                        
                        \node (n\i) [draw, fill=white, minimum size=0cm, inner sep=0mm, circle] at (c.\i*360/3+90) {
                            \scalebox{0.8}{
                                \x
                            }
                        };
                    }
                    \node (nm) [draw, fill=white, minimum size=0cm, inner sep=0mm, circle] at (c.center) {
                        \scalebox{0.8}{
                            \shortstack{
                                \Large \textbf{System} \\ \Large \textbf{Entwicklung} \\
                                \rule{3cm}{1pt} \\
                                \footnotesize
                                Produktentwicklung,\\ 
                                \footnotesize
                                Technologietransfer, \\
                                \footnotesize
                                Prototyping
                            }
                        }
                    };
                    \node [anchor=south east, rectangle, minimum width=4cm, rounded corners, text=white, fill=black, inner sep=1.7mm] at (frame.south east) {
                        \textbf{nach J. Nunamaker}
                    };

                    \draw[exstyle] (nm.200) -- (n1.40);
                    \draw[exstyle] (n1.25) -- (nm.215);

                    \draw[exstyle] (nm.340) -- (n2.140);
                    \draw[exstyle] (n2.155) -- (nm.325);

                    \draw[exstyle] (nm.{83}) -- (n3.{277});
                    \draw[exstyle] (n3.{263}) -- (nm.{97});

                    \draw[exstyle, bend left=41.5](n3.0) to (n2.60);
                    \draw[exstyle, bend left=41.5](n2.240) to (n1.300);
                    \draw[exstyle, bend left=41.5](n1.120) to (n3.180);

                    \draw[exstyle, bend right=41.5](n2.78) to (n3.342);
                    \draw[exstyle, bend right=41.5](n1.318) to (n2.222);
                    \draw[exstyle, bend right=41.5](n3.198) to (n1.102);

                    \node[circle, fill=white, draw, minimum size=1cm] at (n1.71) {1};
                    \node[circle, fill=white, draw, minimum size=1cm] at (n3.25) {2};
                    \node[circle, fill=white, draw, minimum size=1cm] at (nm.45) {3};
                    \node[circle, fill=white, draw, minimum size=1cm] at (n2.110) {4};
                }
            ]
            \end{tcolorbox}
        \end{minipage}
    }
    \caption{Phasen der Problemlösung nach Nunamaker \cite{nunamaker}.}
    \label{sec1:intro:subsec:methodology-goals:fig:nunamaker}
\end{figure}
\begin{itemize}
    \item Detaillierter auf Nunamaker eingehen / Schaubild einfügen.
\end{itemize}

\subsection{Ansatz auf Aufbau der Arbeit}
\label{sec1:intro:subsec:approach-structure}
\begin{itemize}
    \item Im Ansatz fehlt konkrete Zuordnung der FZ zu den einzelnen Kapiteln. Tabelle analog zu Tab.1, die das Umgruppieren verdeutlicht, wäre hilfreich.
\end{itemize}

\subsection{Arbeits- und Zeitplan}
\label{sec1:intro:subsec:work-time-plan}